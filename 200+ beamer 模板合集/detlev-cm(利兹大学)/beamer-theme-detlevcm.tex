
\documentclass[aspectratio=43,display]{beamer}

%% http://tex.stackexchange.com/questions/114219/add-notes-to-latex-beamer
%\documentclass[note]{beamer}
%\setbeameroption{show notes}

\usetheme{DetlevCM}
%% The colourtheme must be provided by the user
%% This allows extra flexibility and allows the use of other
%% colour themes.
\usecolortheme{ETII}



\usepackage{enumerate}
\usepackage{hyperref}
\hypersetup{
bookmarksopen=false,
pdfpagemode=UseNone,
%pdfpagemode=FullScreen,   %% Enable to have Adobe Reader query for fullscreen mode
pdfauthor={Detlev Conrad Mielczarek} %% Enter the apppropriate author in here
}

%% Title
\title[Beamer Theme]{LaTeX Beamer Theme following Uni Leeds, engineering branding}

%% Author with both abbreviation and affiliation
\author[DCM] %
{%
	Detlev C. Mielczarek\inst{1}
}

%% Auther affiliation
\institute[UL]{\inst{1}Your Institution Here}



%% Begin the actual document
\begin{document}


\maketitle



\begin{frame}{Optional Table of Contents Slide}
	\tableofcontents
\end{frame}

%% Sections must be explicitly defined
%% square brackets for the short form in the headline
%% braces for the long entry in the table of contents
\section[Short Title]{Long Title}

%% slide titles must be given explicitly in braces
%% exclude to not print a slide title
%% the slide title can be different from the section
\begin{frame}{Example Frame}
\begin{itemize}[<1->]
	
	\item A LaTeX template designed for use at the University of Leeds \\ (NOT officially endorsed in any way)
	
	\item This is a sample presentation \& manual
	
	\item Other colour choices are allowed in the preamble
	
	\item University logo not supplied (copyright issues) - use your own (LogoTop.png)
	
	\item The images are copyrighted - a licence is herewith granted to use the supplied low resolution version on a royalty free basis in connection with this theme
\end{itemize}
\end{frame}

\section[No Title Frame]{Frame With an Empty Title Title}
\begin{frame}[<1->]
	
	\begin{itemize}
	
		\item frame titles can be included by inserting them in a second set of braces, e.g.: \\
		\textbackslash begin\{frame\}\{frametitle\} \\
		\textbackslash end\{frame\}
		
		\item frame titles can be omitted by excluding the second set of braces, e.g.: \\
		\textbackslash begin\{frame\} \\
		\textbackslash end\{frame\}
		
		\item please refer to the Beamer documentation for Beamer options \\
		\footnotesize
		\url{https://www.ctan.org/tex-archive/macros/latex/contrib/beamer}
	\end{itemize}
	
\end{frame}



\end{document}