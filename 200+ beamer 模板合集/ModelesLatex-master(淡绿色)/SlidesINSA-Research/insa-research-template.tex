\documentclass[slideopt,A4,compress]{beamer}
\usepackage[utf8x]{inputenc}
\usepackage[french]{babel}
\usepackage[OT1]{fontenc}
\usepackage{layout}
\usetheme{default}
\useoutertheme{default}
% Couleurs recherche
\definecolor{insa}{HTML}{98BF00}
\definecolor{insa2}{HTML}{BDD671}
% Couleurs enseignement
% \definecolor{insa}{rgb}{0,169,192}
% \definecolor{insa2}{rgb}{104,197,211}
% \useinnertheme{circles}

% Column snippets (auto compute 2 columns with an easy separation snippet)
\usepackage{calc}
\newlength\colonewidth
\newlength\coltwowidth

%%%%%%%%%%%%%%%%%%%%%%%%%%%%%%%%%%%%%%%%%%%%%%%%%%%%%%
% Draft mode
% Don't draw the background on slides (reduces the
% compilation time by half... TikZ is awesome but a 
% bit heavy on the resources ;-) )
%%%%%%%%%%%%%%%%%%%%%%%%%%%%%%%%%%%%%%%%%%%%%%%%%%%%%%
\newif\iffinal
\finaltrue % Sets the backgrounds and stuff

%%%%%%%%%%%%%%%%%%%%%%%%%%%%%%%%%%%%%%%%%%%%%%%%%%%%%%
% Automatic item spacing
% Increases the separating vspace between items, the 
% default packs too much on a slide
%%%%%%%%%%%%%%%%%%%%%%%%%%%%%%%%%%%%%%%%%%%%%%%%%%%%%%

\usepackage{enumitem}
\setitemize{label=\usebeamerfont*{itemize item}%
  \usebeamercolor[fg]{itemize item}
  \usebeamertemplate{itemize item}}
\setlist[itemize]{label=\usebeamerfont*{itemize item}%
  \usebeamercolor[fg]{itemize item}
  \usebeamertemplate{itemize item},itemsep=10pt}
\setlist[itemize,2]{label=\usebeamerfont*{itemize item}%
  \usebeamercolor[fg]{itemize item}
  \usebeamertemplate{itemize item},itemsep=6pt,topsep=6pt}

%%%%%%%%%%%%%%%%%%%%%%%%%%%%%%%%%%%%%%%%%%%%%%%%%%%%%%%%
% Variable length block (e.g. captions and stuff)
% Usage : 
% \centering or \hspace*{...}
% \begin{varblock}[optional size, defaults at 0.9\textwidth]{Block title}
% ...
% \end{varblock}
%%%%%%%%%%%%%%%%%%%%%%%%%%%%%%%%%%%%%%%%%%%%%%%%%%%%%%%%ù

\newenvironment<>{varblock}[2][.9\textwidth]{%
  \setlength{\textwidth}{#1}
  \begin{minipage}{\textwidth}
  \begin{actionenv}#3
  \def\insertblocktitle{#2}
  \par
  \usebeamertemplate{block begin}
}{%
  \par
  \usebeamertemplate{block end}
  \end{actionenv}
  \end{minipage}
}

% TikZ stuff
\usepackage{tikz}
%% TikZ plots
\usepackage{pgf}
\usepackage{pgfplots}
\usepackage{tikz-3dplot}
\tdplotsetmaincoords{70}{120}
\pgfplotsset{compat=newest} 
\pgfplotsset{plot coordinates/math parser=false}
\newlength\figurewidth
% TikZ libraries
\usetikzlibrary{shapes.geometric}
\usetikzlibrary{shapes.misc}
\usetikzlibrary{decorations}
\usetikzlibrary{shadows}
\usetikzlibrary{calc}
\usetikzlibrary{shadows.blur}
\usetikzlibrary{positioning}
\tikzstyle{itnode}=[circle,thick,draw]
\tikzstyle{na} = [baseline=-.5ex]
\tikzstyle{na2} = [baseline=2ex]
\tikzstyle{every picture}+=[remember picture]
% \tikzset{thick}


\usepackage{amsmath,amssymb}
\usepackage{graphicx}

% Text boxes with absolute positioning
\usepackage[absolute,showboxes,overlay]{textpos}
\TPshowboxesfalse
\textblockorigin{0mm}{0mm}

% Use appendix as complementary slides that do not show in the "total slides"
\usepackage{appendixnumberbeamer}

% Alternative fonts (Utopia and Helvetica for the small caps)
% \usepackage{fourier}
% \usepackage{helvet}
% Use sans-serif
\sffamily

%----------- Custom beamer stuff ------------------
\setbeamertemplate{navigation symbols}{}
%\setbeamersize{text margin left=0.5cm,text margin right=0.5cm,text margin top=1cm}
\setbeamersize{text margin left=0.5cm,text margin right=0.5cm}
% Squared enumerates
\setbeamertemplate{sections/subsections in toc}[square]
\setbeamercolor{item}{fg=insa}

% Frametitle overlay
\setbeamertemplate{frametitle}{\color{insa!50!black} \hspace{5mm} \insertframetitle \vskip-4pt }
\setbeamerfont{frametitle}{size=\large}
% Block colors and shape
\setbeamertemplate{blocks}[rounded][shadow=true]
\setbeamercolor{block title}{fg=white,bg=insa!80!black}
\setbeamercolor{block body}{fg=black,bg=insa2!50}

% Alerts in black and bold
\setbeamerfont{alerted text}{series=\bfseries}
\setbeamercolor{alerted text}{fg=black}

% Smaller subsections in ToC
\setbeamerfont{subsection in toc}{size=\footnotesize}

% Custom beamer citation command, using the beamer article icon
\newcommand\insertbeamercitationlabel{\includegraphics[width=2mm]{beamericonarticle.pdf}}
\newcommand\citeb[2]{
\nocite{#1} {\footnotesize [\raisebox{-1pt}{\insertbeamercitationlabel}\emph{#2}]}
}

% Note page
\setbeameroption{show notes}

% No headline or footline
\setbeamertemplate{footline}{}
\setbeamertemplate{headline}{}

% Math commands
\newcommand{\E}[1]{\mathbb{E}\left[{#1}\right]}
\DeclareMathOperator*{\argmax}{arg\,max}

\begin{document}

%%%%%%%%%%%%%%%%%%%%%%% TITLE PAGE %%%%%%%%%%%%%%%%%%%%%%%
%%%%%%%%%%%%%%% The customization is below %%%%%%%%%%%%%%%
%%%%%%%%%%%%%%%%%%%%%%%%%%%%%%%%%%%%%%%%%%%%%%%%%%%%%%%%%%

\iffinal
\setbeamertemplate{background canvas}{
\begin{tikzpicture}
    \useasboundingbox [fill=insa] (0,0) rectangle (128mm,96mm);
    \node[trapezium, draw,white,,fill=white,trapezium left angle=70, trapezium right angle=102,minimum height = 58mm,minimum width = 28mm,rounded corners,anchor=south] at (-10mm,21mm) {};
    \node[trapezium, draw,white,,fill=white,trapezium left angle=78, trapezium right angle=102,minimum height = 58mm,minimum width = 110mm,rounded corners,anchor=south,top color=insa!10,bottom color=white] at (95mm,42mm) {};
    \fill[white] (130mm,-25mm) ellipse (60mm and 40mm);
    \node [right] at (11mm,81.5mm) {\scriptsize \bfseries \color{insa2} RECHERCHE};
    \node [anchor=west,rotate=78,insa!50] at (20mm,20mm) {\tiny INSTITUT NATIONAL DES SCIENCES APPLIQU\'EES DE LYON};
    \node [right] at (100mm,6mm) {\includegraphics[width=25mm]{insa-couleur2.png}};
%    \node [right] at (0mm,34mm) {\includegraphics[width=19mm]{logo_universite2.png}};
    \node [right, opacity=0.5] at (35mm,80mm) {\includegraphics[width=100mm]{bolochon.png}};
    \node [right,anchor=west] at (4mm,70mm) {\includegraphics[width=20mm]{citi_autrevert.pdf}};
%    \node [right,anchor=west] at (2mm,52mm) {\includegraphics[width=20mm]{inria_logo.pdf}};
    \node [right,anchor=west] at (2mm,52mm) {\includegraphics[width=19mm]{logo_universite2.png}};
    \node [trapezium,trapezium left angle=78,trapezium right angle = 102,anchor=north west,text width = 92mm,text height=12mm,white,fill=insa!70,rounded corners] at (66.8mm,39.8mm) {};
\end{tikzpicture}
} 
\else
\setbeamertemplate{background canvas}{
\begin{tikzpicture}
    \useasboundingbox [fill=insa] (0,0) rectangle (128mm,96mm);
\end{tikzpicture}
} 
\fi

\begin{frame}


\begin{textblock*}{10cm}(28mm,58mm)
{\textcolor{white} {
%% CUSTOMIZATION :  Titre sur 2 lignes
{\Large \bfseries \hspace{2mm} A nice title that, at most, is spanning}\\[1mm]
{\Large \bfseries 2 (!) lines}\\[2mm]
  %% CUSTOMIZATION : Sous-titre
   	{\footnotesize \hspace{1.5cm} \emph{Subtitle}} \\
    {\footnotesize \hspace{2.4cm} \emph{Something better than hspaces may be needed}}
	}
  }
	\end{textblock*}
\vspace*{-4pt}
\end{frame}

%%%%%%%%%%%%%%%%%%%%%%% GENERIC PAGE %%%%%%%%%%%%%%%%%%%%%%%

\iffinal
\setbeamertemplate{background canvas}{\begin{tikzpicture}
    \useasboundingbox [fill=insa] (0,0) rectangle (\paperwidth,\paperheight);
    \fill [white,rounded corners] (0,6mm) rectangle (128mm,87mm); 
    \node [trapezium,rounded corners,text width=15mm,text height=6mm,text centered,color=insa2!50!black,fill=insa2!50,trapezium left angle=78,trapezium right angle = 102,anchor=south,blur shadow={shadow blur steps=8,shadow blur radius=1mm,shadow xshift=-0pt,shadow yshift=-1pt}] at (121mm,-1mm) {\raisebox{2mm}{\footnotesize \bfseries \insertpagenumber\,/\,\insertpresentationendpage}}; 
    \node [trapezium,rounded corners,text width=15mm,text height=6mm,text centered, fill=insa2!50,trapezium left angle=102,trapezium right angle = 78,anchor=south,blur shadow={shadow blur steps=8,shadow blur radius=1mm,shadow xshift=-0pt,shadow yshift=-1pt}] at (9mm,-1mm) {\raisebox{1mm}{\includegraphics[width=15mm]{insa-couleur2.png}}};
    \node [trapezium,rounded corners,text width=115mm,text height=8mm,fill=insa2!50,trapezium left angle=102,trapezium right angle = 102,anchor=south,blur shadow={shadow blur steps=8,shadow blur radius=1mm,shadow xshift=-0pt,shadow yshift=0pt}] at (64mm,88mm) {};
    %% CUSTOMIZATION : Texte de la ligne du bas
    \node [insa!40] at (64mm,3mm) {\tiny \textsc{\bfseries Footline}};
\end{tikzpicture}} 
\else
\setbeamertemplate{background canvas}{
  \begin{tikzpicture}
    \useasboundingbox [fill=white] (0,0) rectangle (\paperwidth,\paperheight);
    \node [insa!40] at (64mm,3mm) {\tiny \textsc{\bfseries Footline}};
    \node [color=insa2!50!black] at (121mm,2mm) {\footnotesize \bfseries \insertpagenumber\,/\,\insertpresentationendpage};
    \draw[insa] (0,6mm) -- (\paperwidth,6mm);
    \draw[insa] (0,87mm) -- (\paperwidth,87mm);
  \end{tikzpicture}
}
\fi

%%%%%%%%%%%%%%%%%%%%%% Your stuff begins here! %%%%%%%%%%%%%%%%%%%%%%

\begin{frame}
\frametitle{Outline of the talk}
\tableofcontents[hidesubsections]
\end{frame}

\section{Section 1}
\subsection{Subsection 1}
\begin{frame}{\insertsubsectionhead}
  A dummy frame with stuff in it. \citeb{Beamer2013}{Beamer citation icon is awesome}
  \begin{itemize}
  \item Blablabla
    \begin{itemize}
      \item \alert{Alert text}
    \end{itemize}
  \end{itemize}
  \begin{block}{Block title}
  A block
  \end{block}
  \centering
  \begin{varblock}[0.6\textwidth]{Block title}
  A variable length block
  \end{varblock}
\end{frame}

% Insert a "ToC" frame when calling \section{...} with subsections detailed
\AtBeginSection{
  \begin{frame}\frametitle{Outline of the talk}
  \tableofcontents[currentsection,hideothersubsections]
  \end{frame}
}

\section{Section 2}
\subsection{Subsection 2}
\begin{frame}{\insertsubsectionhead}
  A dummy frame with stuff in it. \citeb{Beamer2013}{Beamer citation icon is awesome}
  \begin{itemize}
  \item Blablabla
    \begin{itemize}
      \item \alert{Alert text}
    \end{itemize}
  \end{itemize}
  \begin{block}{Block title}
  A block
  \end{block}
  \centering
  \begin{varblock}[0.6\textwidth]{Block title}
  A variable length block
  \end{varblock}
\end{frame}

\end{document}


