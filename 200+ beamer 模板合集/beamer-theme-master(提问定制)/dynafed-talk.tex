\documentclass{beamer}
\usepackage[utf8]{inputenc}
\usepackage[T1]{fontenc}
\usepackage{hyperref}
\usepackage{graphicx}
\title{Update on the Dynamic Federation}
\date[ISPN ’80]{Technical Coordination Board}
\author[Euclid]{Frank Berghaus \href{mailto:berghaus@cern.ch}{\texttt{berghaus@cern.ch}}}

% --- Blocks ---
\setbeamertemplate{blocks}[default]

\usetheme{frank}

\begin{document}

\begin{frame}
\titlepage
\end{frame}


\begin{frame}
\frametitle{Review: What does DynaFed do?}
\begin{itemize}
\item Aggregates storage and metadata farms on-the-fly
\item Exposes standard protocols that support redirections and WAN data access
\item Creates (the illusion of) a unique namespace from a set of distinct storage or metadata endpoints
\item read and write support
\end{itemize}
\end{frame}

\begin{frame}
  \frametitle{Dynafed Namespace}
  \begin{figure}
      \centering
      \includegraphics[width=\columnwidth]{dynafed-namespaces.png}
  \end{figure}
\end{frame}

\begin{frame}
  \frametitle{Motivation}
  \begin{itemize}
    \item Ease of negotiating connections between storage and users/jobs
    \item Integrate multiple storage backend under the same protocol
    \item Stability because of multiplication of data sources
    \item Ease the pain of supporting SRM
    \item What have I forgotten?
  \end{itemize}
\end{frame}

\begin{frame}
  \frametitle{ATLAS Projects with DynaFed}
  \begin{itemize}
    \item UVic in Victoria: Rolf Seuster \& Marcus Ebert
    \item UVic at CERN: Frank Berghaus
    \item At RAL: Alastair Dewhurst
    \item Italy in Frascati, Napoli and Roma: Alessandro De Salvo
  \end{itemize}
  With lots of help from ADC (Ale and Ivan) and DDM (Mario and Cedric)!
\end{frame}

\begin{frame}
  \frametitle{Status of DynaFed at CERN}
  \begin{itemize}
    \item Endpoint: \url{https://dynafed-dev.cern.ch/atlas/scratchdisk/}
    \item Panda Queue: CERN-EXTENSION\_MCORE
		\item DDM Endpoint: CERN-EXTENSION\_SCRATCHDISK
    \item Setup:
    \begin{itemize}
      \item DynaFed running on CC7 on CERN OpenStack
      \item Linked to CephS3 at CERN
    \end{itemize}
		\item Authentication:
    \begin{itemize}
      \item Using X.509 with VOMS extensions
      \item Federation forwards signed URLs to authenticated users (signatures last 1h)
    \end{itemize}
    \item Jobs write to DynaFed
    \item Rucio replication has errors with protocol mismatches
  \end{itemize}
\end{frame}

\begin{frame}
  \frametitle{Status of DynaFed at RAL}
  \begin{itemize}
    \item Endpoint: \url{https://dynafed.stfc.ac.uk/gridpp/}
		\item DDM Endpoint: RAL-AZURE\_DATADISK, RAL-AZURE\_SCRATCHDISK
    \item Setup:
    \begin{itemize}
      \item Single VM but easily scalable
      \item Linked to S3 Gateway at RAL
    \end{itemize}
		\item Authentication:
    \begin{itemize}
      \item Using X.509 a python plugin with gridmap - allow in browser identification
      \item Federation forwards signed URLs to authenticated users (signatures last 1h)
    \end{itemize}
    \item Adding support for gsiftp - work in progress
    \item Introduction by Alastair: \url{https://indico.cern.ch/event/602119/}
  \end{itemize}
\end{frame}

\begin{frame}
  \frametitle{Status of DynaFed in Italy}
  \begin{itemize}
    \item Endpoint: \url{https://atlas-dynafed.roma1.infn.it/infn/}
		\item DDM Endpoint:
    \item Setup:
    \begin{itemize}
      \item Frascati, Napoli and Roma DPM  production storages
      \item S3/WebDav test storage area in Roma (160TB)
    \end{itemize}
		\item Testing Plans:
    \begin{itemize}
      \item Complete the endpoint commissioning and attach it to a new Panda Site
      \item Test the performance and scalability of the DynaFed setup with real jobs
      \item Compare the performance of the same analysis, at different concurrency levels
    \end{itemize}
  \end{itemize}
\end{frame}

\begin{frame}
  \frametitle{Status of DynaFed at UVic}
  \begin{itemize}
    \item Endpoint: \url{https://dynafed.heprc.uvic.ca/fed/}
    \item Setup:
    \begin{itemize}
      \item DynaFed running on SL6 VM
      \item Compute Canada S3 storage
    \end{itemize}
		\item Developing integration with Belle-II/DIRAC
    \item Will add UVic DynaFed to IAAS Panda Queue using CERN example as graft
  \end{itemize}
\end{frame}

\begin{frame}
  \frametitle{Problem: Directories/Collections}
  \begin{block}{WebDAV RFC 8.7.2 PUT for Collections}
    When the PUT operation creates a new non-collection resource all ancestors MUST already exist. If all ancestors do not exist, the method MUST fail with a 409 (Conflict) status code. For example, if resource /a/b/c/d.html is to be created and /a/b/c/ does not exist, then the request must fail.
  \end{block}
  \begin{itemize}
    \item<1- > Default Rucio HTTP(S) behaviour: make all parent directory before putting file
    \item<2- > In object stores directories/collections do not exist
    \item<2- > MKCOL not implemented in federation
  \end{itemize}
\end{frame}

\begin{frame}
  \frametitle{Solution: Directories/Collections}
  \begin{itemize}
    \item Rucio: Resolve DynaFed and WebDAV standard behaviour:
    \begin{itemize}
      \item Flag HTTP resources that do not implement full WebDAV standard
      \item Until flag supported in Rucio and AGIS relying on \alert{hack in Rucio}
    \end{itemize}
    \item DynaFed: Create union of WebDAV standard and object store behaviour in DynaFed: \href{https://its.cern.ch/jira/browse/LCGDM-2373}{LCGDM-2373}
    \item \emph{Note}: Avoid combining writable WebDAV and object stores in a DynaFed for now
  \end{itemize}
\end{frame}

\begin{frame}
  \frametitle{Problem: Moving Data to DynaFed}
  \begin{itemize}
    \item DynaFed is HTTP only (no gsiftp)
    \item Protocol mismatch when trying to copy input datasets for AFT/PFT to DynaFed
    \item Alastair is implementing gsiftp support at RAL
    \item ~70\% of DDMEndpoints support HTTP/DAV why is this a problem?
  \end{itemize}
\end{frame}

\begin{frame}
  \frametitle{Plans and Goals}
  \begin{enumerate}
    \item Replication rules from any grid storage to and from dynafed
    \item Get jobs running which read from EOS and write to DynaFed
    \item Get jobs running which read from DynaFed and write to EOS
    \item Get jobs running which read from DynaFed and write to DynaFed
  \end{enumerate}
  \begin{block}{Big Picture}
    This could allow us to implement tactical storage transparently for PanDA and Rucio
  \end{block}
\end{frame}

\end{document}
