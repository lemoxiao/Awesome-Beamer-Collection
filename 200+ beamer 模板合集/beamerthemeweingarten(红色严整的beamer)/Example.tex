%\documentclass[handout]{beamer}
\documentclass{beamer}
\usepackage[ngerman]{babel}

\usepackage[utf8]{inputenc}

\usetheme{Weingarten}


\setbeamercovered{transparent}

%\setbeamertemplate{footline}[frame number]

\title{Weingarten}
\subtitle{A theme for \LaTeX - Beamer}

\author{Your name}
\institute{Your institute}
\date{Januar 2009}

\begin{document}
%
% Titlepage
%

\begin{frame}
	\titlepage
\end{frame}

% 
% TOC 
%
\section*{Outline}
\begin{frame}{Outline}
	\tableofcontents
\end{frame}

\section{The theme "`Weingarten"'}

\subsection{General things}
\begin{frame}{Weingarten}{Some facts about the theme}
Something about the theme "Weingarten":
\begin{itemize}
	\item It's under the GPL, so you can modify it by your own.
	\item The colors which are used are the colors of the Weingarten (http://www.weingarten-online.de/), the city where I go to school.
	\item The theme is called "`Weingarten"', because the themes of \LaTeX-Beamer are always named after some cities - and this is the city where my "institute" is :)
	\item Have a lot of fun with this theme!
\end{itemize}
\end{frame}

\subsection{What you can do}
\begin{frame}{The title of your frame}{And the Subtitle}
	\begin{block}{You can use blocks to structure your topic}
	Here's your very interesting content :)
	\end{block}

\end{frame}

\end{document}