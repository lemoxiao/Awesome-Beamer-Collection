\documentclass[english,sectioncirclenumberstyle]{le2iutbmbeamer}

\usepackage[utf8]{inputenc}
\usepackage{algpseudocode}
\usepackage{multicol}
\usepackage{tabularx}

\graphicspath{{./docfigs/}}

\title{Documentation of the LE2I-UTBM Theme for Beamer}
\subtitle{Version \insertleiiutbmbeamerthemeversion}

\author[S.~Galland]{St\'ephane~Galland}

\keywords{Beamer; Theme; LE2I; UTBM; Documentation}
\subject{Beamer Theme for LE2I Laboratory at UTBM}

\finalslidetext{End of the Documentation}

\usefootlinewithsections

\pgfdeclareimage[height=.2cm]{fakele2ilogoinpartners}{le2ilogo}
\pgfdeclareimage[height=.2cm]{fakeutbmlogoinpartners}{utbmlogo}
\pgfdeclareimage[height=.2cm]{fakeubfclogoinpartners}{ubfclogo}
\pgfdeclareimage[height=.2cm]{fakecnrslogoinpartners}{cnrslogo}
\pgfdeclareimage[height=.25cm]{fakele2ilogointitle}{le2ilogoinv}
\pgfdeclareimage[height=.25cm]{fakemaglogointitle}{multiagentlogoinv}
\pgfdeclareimage[height=.25cm]{fakeutbmlogointitle}{utbmlogoinv}
\pgfdeclareimage[height=.25cm]{fakeubfclogointitle}{ubfclogoinv}
\pgfdeclareimage[height=.25cm]{fakecnrslogointitle}{cnrslogoinv}
\pgfdeclareimage[height=.25cm]{fakeuserlogointitle}{utbmlogo}

\makeatletter
\let\fakebackground\beamer@theme@leiiutbm@outer@background
\makeatother

\makeatletter
\let\fakeoldunderline\beamer@theme@leiiutbm@oldunderline
\makeatother

\newcommand{\fakeslide}[3][fakele2ilogointitle]{
	\framebox{\begin{minipage}{.4\paperwidth}
		\begin{picture}(0,90)
			\put(-3,81.25){\scalebox{0.41}{\fakebackground}}
			\put(10,86){\textcolor{white}{\tiny #2}}
			\ifthenelse{\equal{a#1}{a}}{}{\put(-1,84.5){\pgfuseimage{#1}}}
			#3				
			\put(100,-1){
				\pgfuseimage{fakele2ilogoinpartners}\hspace{0.05cm}\pgfuseimage{fakeutbmlogoinpartners}\hspace{0.05cm}\pgfuseimage{fakeubfclogoinpartners}\hspace{0.05cm}\pgfuseimage{fakecnrslogoinpartners}}
		\end{picture}
	\end{minipage}}
}

\begin{document}

%%%%%%%%%%%%%%%%%%%%%%%%%%%%%%%%%%%%%%%%%%%%%%%%%
\begin{frame}{Introduction}
	\begin{itemize}
	\item This document explains the {\LaTeX} macros that are provided by the LE2I-UTBM theme for beamer.
	\vfill
	\item To use the LE2I-UTBM theme for beamer, you must:
		\begin{enumerate}
		\item Install the files of the LE2I-UTBM theme inside your {\LaTeX} distribution.
		\item Create a document, which uses the document class: \texttt{le2iutbmbeamer}
		\end{enumerate}
	\end{itemize}
\end{frame}

\begin{frame}{Major Features}
	\begin{itemize}
	\item Provide a presentation style related to the research group of the LE2I located at the UTBM.
	\vfill
	\item Add a title slide (with label \texttt{titleslide} for \texttt{{\textbackslash}hyperlink}) as the first slide.
	\vfill
	\item Add a final slide at the end of the presentation to avoid ``black'' screen at the end of the presentation.
	\vfill
	\item Provide high level macros for building the slides.
	\end{itemize}
\end{frame}

\begin{frame}{Read the Beamer Documentation}
	\huge
	\alertbox{It is recommended to read the user guide of Beamer for obtain a view of the available macros.}
	\vspace{2em}
	\alertbox*{\url{http://texdoc.net/texmf-dist/doc/latex/beamer/doc/beameruserguide.pdf}}
\end{frame}

%%%%%%%%%%%%%%%%%%%%%%%%%%%%%%%%%%%%%%%%%%%%%%%%%
\tableofcontentslide

%%%%%%%%%%%%%%%%%%%%%%%%%%%%%%%%%%%%%%%%%%%%%%%%%
\section{Predefined Values}
\tableofcontentslide[sectionstyle={show/shaded},subsectionstyle={show/show/hide},subsubsectionstyle={hide/hide/hide/hide}]

\subsection{Colors}

\begin{frame}{Colors}
	\begin{itemize}
	\item The LE2I-UTBM theme defines the colors for the macros provided by the \texttt{xcolor} package:
		\begin{itemize}
		\item \texttt{LE2IUTBMblue} : \fbox{\textcolor{LE2IUTBMblue}{Example}} \colorbox{LE2IUTBMblue}{\textcolor{white}{Example}}
		\vfill
		\item \texttt{LE2IUTBMmagenta} : \fbox{\textcolor{LE2IUTBMmagenta}{Example}} \colorbox{LE2IUTBMmagenta}{\textcolor{white}{Example}}
		\vfill
		\item \texttt{LE2IUTBMgreen} : \fbox{\textcolor{LE2IUTBMgreen}{Example}} \colorbox{LE2IUTBMgreen}{\textcolor{white}{Example}}
		\vfill
		\item \texttt{LE2IUTBMdarkblue} : \fbox{\textcolor{LE2IUTBMdarkblue}{Example}} \colorbox{LE2IUTBMdarkblue}{\textcolor{white}{Example}}
		\vfill
		\item \texttt{LE2IUTBMlightblue} : \fbox{\textcolor{LE2IUTBMlightblue}{Example}} \colorbox{LE2IUTBMlightblue}{\textcolor{white}{Example}}
		\vfill
		\item \texttt{LE2IUTBMlightestblue} : \fbox{\textcolor{LE2IUTBMlightestblue}{Example}} \colorbox{LE2IUTBMlightestblue}{\textcolor{black}{Example}}
		\vfill
		\item \texttt{LE2IUTBMdarkmagenta} : \fbox{\textcolor{LE2IUTBMdarkmagenta}{Example}} \colorbox{LE2IUTBMdarkmagenta}{\textcolor{white}{Example}}
		\vfill
		\item \texttt{LE2IUTBMlightmagenta} : \fbox{\textcolor{LE2IUTBMlightmagenta}{Example}} \colorbox{LE2IUTBMlightmagenta}{\textcolor{white}{Example}}
		\vfill
		\item \texttt{LE2IUTBMlightestmagenta} : \fbox{\textcolor{LE2IUTBMlightestmagenta}{Example}} \colorbox{LE2IUTBMlightestmagenta}{\textcolor{black}{Example}}
		\vfill
		\item \texttt{LE2IUTBMdarkgray} : \fbox{\textcolor{LE2IUTBMdarkgray}{Example}} \colorbox{LE2IUTBMdarkgray}{\textcolor{white}{Example}}
		\vfill
		\item \texttt{LE2IUTBMlightgray} : \fbox{\textcolor{LE2IUTBMlightgray}{Example}} \colorbox{LE2IUTBMlightgray}{\textcolor{black}{Example}}
		\end{itemize}
	\end{itemize}
\end{frame}

\begin{frame}{Beamer Colors}
	\begin{itemize}
	\item The LE2I-UTBM theme defines color for beamer, that may be used with the macro \texttt{{\textbackslash}usebeamercolor\{color\_name\}}:
		\begin{itemize}
		\item \texttt{code keyword} : \fbox{\usebeamercolor[fg]{code keyword}Example}
		\vfill
		\item \texttt{code string} : \fbox{\usebeamercolor[fg]{code string}Example}
		\vfill
		\item \texttt{code comment} : \fbox{\usebeamercolor[fg]{code comment}Example}
		\end{itemize}
	\end{itemize}
\end{frame}

\subsection{Styles}
\begin{frame}{Beamer Templates}
	\begin{itemize}
	\item The LE2I-UTBM theme defines the specific templates for Beamer. These templates could be used with 
		\texttt{{\textbackslash}usebeamertemplate\{name\}}:
		\begin{itemize}
		\item \texttt{code basic style} : the style for the standard text in a program or an algorithm. 
		\vfill
		\item \texttt{code inline style} : the style for the standard text in a program or an algorithm when it is displayed inline. 
		\vfill
		\item \texttt{code identifier style} : the style for the identifiers in a program or an algorithm. 
		\vfill
		\item \texttt{code keyword style} : the style for the keywords in a program or an algorithm. 
		\end{itemize}
	\end{itemize}
\end{frame}

%%%%%%%%%%%%%%%%%%%%%%%%%%%%%%%%%%%%%%%%%%%%%%%%%
\section{Class Options and Document Declaration}
\tableofcontentslide[sectionstyle={show/shaded},subsectionstyle={show/show/hide},subsubsectionstyle={hide/hide/hide/hide},sections={1-6}]

\subsection{Class Options}

\begin{frame}[t]{Class Options}
	The document class \texttt{le2iutbmbeamer} supports the following options.
	\begin{block}{Presentation language}
		\begin{itemize}
		\item \texttt{english}: the slides are written in English (this is the default).
		\item \texttt{french}: the slides are written in French.
		\end{itemize}
	\end{block}
	\begin{block}{Automatic slides}
		\begin{itemize}
		\item \texttt{nocover}: do not add a slide for the title and the ``thanks''.
		\item \texttt{thanksslide}: add a slide with the word ``Thanks'' at the end (this is the default). 
		\item \texttt{questionslide}: add a slide with the word ``Question'' at the end. 
		\item \texttt{repeattitleslide}: add the title slide at the end.
		\end{itemize}
	\end{block}
\end{frame}

\begin{frame}[t]{Class Options \insertcontinuationtext}
	\begin{block}{Slide numbering}
		\begin{itemize}
		\item \texttt{textnumberstyle}: the frame numbers are output as text.\hfill\hyperlink{progressbartypes}{\beamergotobutton{See example}}
		\item \texttt{circlenumberstyle}: the frame numbers are output in a circle.\hfill\hyperlink{progressbartypes}{\beamergotobutton{See example}}
		\item \texttt{sectioncirclenumberstyle}: the frame numbers are output in a circle.\hfill\hyperlink{progressbartypes}{\beamergotobutton{See example}}
		\item \texttt{partcirclenumberstyle}: similar to \texttt{circlenumberstyle} except that the indicator is reset at the start of each part.
		\item \texttt{partsectioncirclenumberstyle}: similar to \texttt{sectioncirclenumberstyle} except that the indicator is reset at the start of each part.
		\item \texttt{nonumberstyle}: the frame numbers are not output.
		\end{itemize}
	\end{block}
\end{frame}

\begin{frame}[t]{Class Options \insertcontinuationtext}
	\begin{block}{Footline}
		\begin{itemize}
		\item \texttt{lablogo} and \texttt{nolablogo}: enable or disable the laboratory logos in the foot line.
		\end{itemize}
	\end{block}
	\begin{block}{Event name in specific locations}
		\begin{itemize}
		\item \texttt{eventintitlebar}: on the title slide, the event's name should be put in the title bar.
		\item \texttt{eventbelowauthors}: on the title slide, the event's name should be put below the authors.
		\end{itemize}
	\end{block}
	\begin{block}{Hyphenation}
		\begin{itemize}
		\item \texttt{hyphenation} and \texttt{nohyphenation}: enable or disable the hyphenation of the text on the slides.
		\end{itemize}
	\end{block}
\end{frame}

\begin{frame}[t]{Class Options \insertcontinuationtext}
	\begin{block}{Handout --- Free space beside slides for hand notes}
		\begin{itemize}
		\item \texttt{handout} and \texttt{nohandout}: enable or disable the generation of the handout document.
		\end{itemize}
	\end{block}
	\alertbox{All the options accepted by Beamer are also accepted.}
\end{frame}

\subsection{Title and Subtitle}

\begin{frame}{Title and Subtitle}
	\begin{block}{Title}
		Use the macro: \texttt{{\textbackslash}title\{title of the document\}}
	\end{block}
	\vfill
	\begin{block}{Subtitle}
		Use the macro: \texttt{{\textbackslash}subtitle\{subtitle of the document\}}
	\end{block}
	\vfill
	\alertbox{These macros must be put in the preamble of your document.}
\end{frame}

\subsection{Authors}

\begin{frame}{Authors}
	\begin{itemize}
	\item Authors are defined with the macro: \\
		\texttt{{\textbackslash}author[short]\{long\}}
		\begin{itemize}
		\item the \texttt{long} list of authors is displayed on the front page. You should separate the names with the macro \texttt{{\textbackslash}and}.
		\item the \texttt{short} list of authors is displayed inside the foot line of the slides. You \alert{must not separate} the names with the macro \texttt{{\textbackslash}and}.
		\end{itemize}
	\vfill
	\item \alert{Alternatively}, you could define the authors with the macros: \\
		\texttt{{\textbackslash}addauthor[short]\{long\}} \\
		\texttt{{\textbackslash}addauthor*[short]\{long\}}
		\begin{itemize}
		\item Add \textbf{one} author to the list of the authors.
		\item The ``starred'' version applies some visual indicators to the name (underline, etc.) It may be used to indicate the name of the talker for example.
		\end{itemize}
	\end{itemize}
	\vfill
	\alertbox{These macros must be put in the preamble of your document.}
\end{frame}

\begin{frame}{Author Description}
	\begin{itemize}
	\item Authors may be described with additional information (affiliation...): \\
		\texttt{{\textbackslash}authordescription\{description\}}
	\end{itemize}
	\vfill
	\alertbox{This macro must be put in the preamble of your document.}
\end{frame}

\subsection{Keywords, Subject and Abstract}

\begin{frame}{Keywords of the Document}
	\begin{itemize}
	\item You could associate keywords to the document with: \\
		\texttt{{\textbackslash}keywords\{text\}}
	\vspace{1em}
	\item These keywords are automatically put in the properties of the generated PDF file.
	\vspace{1em}
	\item To insert the keywords into your slides, you could use the macro: \\
			\texttt{{\textbackslash}insertkeywords}
	\end{itemize}
	\begin{example}
		\insertkeywords
	\end{example}
\end{frame}

\begin{frame}{Subject of the Document}
	\begin{itemize}
	\item You could associate a subject to the document with: \\
		\texttt{{\textbackslash}subject\{text\}}
	\vspace{1em}
	\item This subject is automatically put in the properties of the generated PDF file.
	\vspace{1em}
	\item To insert the subject into your slides, you could use the macro: \\
			\texttt{{\textbackslash}insertsubject}
	\end{itemize}
	\begin{example}
		\insertsubject
	\end{example}
\end{frame}

\begin{frame}{Abstract of the Document}
	\begin{itemize}
	\item You could put a short summary of the document in the following environment:
		\texttt{{\textbackslash}begin\{abstract\}} \\
		\texttt{This is a summary.} \\
		\texttt{{\textbackslash}end\{abstract\}}
	\vspace{1em}
	\begin{abstract}
		This is a summary.
	\end{abstract}
	\item If you include an abstract, be sure that it is not some long text but just a very short message.
	\end{itemize}
\end{frame}

\subsection{Name of the Event}

\begin{frame}[t]{Name of the Event}
	\begin{block}{Definition of the event name}
		\begin{itemize}
		\item You could specify the name of the event for which the slides are written: \\
			\texttt{{\textbackslash}event[short]\{full\}}
		\item the \texttt{full} name is displayed on the front page.
		\item the \texttt{short} name is displayed inside the foot line of the slides.
		\item Put these macros into the document preamble.
		\end{itemize}
	\end{block}
	\begin{block}{Insert the event name in your slides}
		\begin{itemize}
		\item You could insert the full name of the event with: \\
			\texttt{{\textbackslash}inserteventname}
		\item You could insert the short name of the event with: \\
			\texttt{{\textbackslash}insertshorteventname}
		\end{itemize}
	\end{block}
\end{frame}

\subsection{Name and URL of the Institute}
\begin{frame}{Institute Name}
	\begin{itemize}
	\item You could change the name of the institute with the following macro in the document preamble: \\
		\texttt{{\textbackslash}institute[short]\{full\}}
	\item the \texttt{full} name is displayed on the front page.
	\item the \texttt{short} name is displayed inside the foot line of the slides.
	\vspace{1em}
	\item The default full name is: \\
		\small\textbf{Universit\'e de Bourgogne Franche-Comt\'e - UTBM - LE2I}, \\
		\small90010 Belfort cedex, France - \url{http://le2i.cnrs.fr}
	\item The default short name is: LE2I-UTBM.
	\end{itemize}
\end{frame}

\begin{frame}{Institute URL}
	\begin{itemize}
	\item You could change the URL of the institute with: \\
		\texttt{{\textbackslash}instituteurl\{url\}}
	\vspace{1em}
	\item The default url is: \url{http://le2i.cnrs.fr}
	\vspace{1em}
	\item To insert the institute's URL, you could use the macro: \\
			\texttt{{\textbackslash}insertinstituteurl}
	\end{itemize}
\end{frame}

%%%%%%%%%%%%%%%%%%%%%%%%%%%%%%%%%%%%%%%%%%%%%%%%%
\section{Header and Footer Tuning}
\tableofcontentslide[sectionstyle={show/shaded},subsectionstyle={show/show/hide},subsubsectionstyle={hide/hide/hide/hide},sections={1-6}]

\subsection{Headline}

\begin{frame}{Headline}
	\begin{itemize}
	\item LE2I-UTBM theme provides an headline in which you can change the logo.
	\vfill
	\item You could change this headline with: \begin{itemize}\footnotesize
		\item \texttt{{\textbackslash}useheaderempty}: the headline is empty (no icon).
		\item \texttt{{\textbackslash}useheaderdefault}: the headline is filled with the default value (may be changed with one of the macros on the following slide).
		\end{itemize}
	\end{itemize}
\end{frame}

\begin{frame}{Headline without Logo}
	\begin{itemize}
	\item \texttt{{\textbackslash}useheaderempty}
	\end{itemize}
	\begin{center}
		\fakeslide[]{XyZ}{}
	\end{center}
\end{frame}

\begin{frame}{Headline with the Multiagent Group Logo}
	\begin{itemize}
	\item \texttt{{\textbackslash}useheaderlinewithmaglogo}: the default headline contains the Multiagent Group logo.
	\end{itemize}
	\begin{center}
		\fakeslide[fakemaglogointitle]{XyZ}{}
	\end{center}
\end{frame}

\begin{frame}{Headline with the LE2I Logo}
	\begin{itemize}
	\item \texttt{{\textbackslash}useheaderlinewithleiilogo}: the default headline contains the LE2I logo.
	\end{itemize}
	\begin{center}
		\fakeslide[fakele2ilogointitle]{XyZ}{}
	\end{center}
\end{frame}

\begin{frame}{Headline with the UTBM Logo}
	\begin{itemize}
	\item \texttt{{\textbackslash}useheaderlinewithutbmlogo}: the default headline contains the UTBM logo.
	\end{itemize}
	\begin{center}
		\fakeslide[fakeutbmlogointitle]{XyZ}{}
	\end{center}
\end{frame}

\begin{frame}{Headline with the UBFC Logo}
	\begin{itemize}
	\item \texttt{{\textbackslash}useheaderlinewithubfclogo}: the default headline contains the UBFC logo.
	\end{itemize}
	\begin{center}
		\fakeslide[fakeubfclogointitle]{XyZ}{}
	\end{center}
\end{frame}

\begin{frame}{Headline with the CNRS Logo}
	\begin{itemize}
	\item \texttt{{\textbackslash}useheaderlinewithcnrslogo}: the default headline contains the CNRS logo.
	\end{itemize}
	\begin{center}
		\fakeslide[fakecnrslogointitle]{XyZ}{}
	\end{center}
\end{frame}

\begin{frame}[t]{Headline with a User-defined Logo}
	\begin{itemize}
	\item \texttt{{\textbackslash}useheaderlinewithuserlogo[options]\{filename\}}: the default headline contains the given image.
	\item Options may be: \begin{itemize}
			\item \texttt{width=<length>} - the width of the image (default: 0.7cm);
			\item \texttt{height=<length>} - the height of the image (no default);
			\item \texttt{x=<float>} - the position of the image along x axis (default: 2);
			\item \texttt{y=<float>} - the position of the image along y axis (default: -22).
			\end{itemize}
	\end{itemize}
	\begin{center}
		\fakeslide[fakeuserlogointitle]{XyZ}{}
	\end{center}
\end{frame}

\subsection{Footline}

\begin{frame}[t]{Footline}
	\begin{itemize}
	\item Beamer provides an footline in which the progress of the presentation may be shown.
	In the LE2I-UTBM theme, this footline is located at the bottom left of the slides.
	\vspace{1em}
	\item You could change this footline with: \begin{itemize}
		\item \texttt{{\textbackslash}usefootlinewithdocumentname}: the footline contains the title of the presentation, and other document informations.
		\item \texttt{{\textbackslash}usefootlinewithsections}: the footline contains the list of the sections of the presentation.
		\end{itemize}
	\vspace{1em}
	\item The following macro is used for inserting the official laboratory logos at the bottom right corner of the slides: \\
		\texttt{{\textbackslash}insertinstitutelogosinfootline\{macros\}}
		\begin{itemize}
		\item The parameter is the set of macros to insert between the logos (basically a spacing macro).
		\item You could redefine this macro for changing the logos.
		\end{itemize}
	\end{itemize}
\end{frame}

\subsection{Partner Logo}

\begin{frame}{Logos for the Partners}
	\begin{itemize}
	\item You could add on all slides one or more logos for your partner(s): \\
		\texttt{{\textbackslash}partnerlogo[options]\{filename\}}
	\item You must call the previous macro for each partner logo.
	\item The \texttt{filename} is the name of the picture.
	\vspace{1em}
	\item This figure is declared with \texttt{{\textbackslash}pgfdeclareimage} with the key ``LE2IUTBMpartnerlogo''.
	\item The figure could be re-used with \texttt{{\textbackslash}pgfuseimage}.
	\item The \texttt{options} are passed to \texttt{{\textbackslash}pgfdeclareimage}. The default option is \texttt{height=.5cm}.
	\item For removing all the partner logos, use: \texttt{{\textbackslash}nopartnerllogo}.
	\end{itemize}
\end{frame}



\subsection{Frame Numbering}

\begin{frame}{Total Number of Frames}
	\begin{itemize}
	\item The total number of slides in the core part of the presentation could be obtained with: \\
		\texttt{{\textbackslash}inserttotalcoreframenumber}
	\end{itemize}
\end{frame}

\begin{frame}[label=progressbartypes,t]{Style for the Frame Numbers}
	\begin{itemize}
	\item The following macro changes the style of the frame numbers:
		\begin{itemize}
		\item \texttt{{\textbackslash}insertframenumbering[type number]}
		\item The \texttt{type number} is the identifier of the progress bar to be used:
		\end{itemize}
	\end{itemize}
	\begin{tabularx}{\linewidth}{|c|c|X|}
	\hline
	\tabularheading\chead{Type number} & \chead{Output} & \chead{Explanation} \\
	\hline
	\texttt{1} & \insertframenumbering[1] & Show the current frame and its position (in blue) in the total number of frames. \\
	\hline
	\texttt{2} & \colorbox{LE2IUTBMdarkgray}{\tiny\insertframenumbering[2]} & Show the current frame and the total number of frames. \\
	\hline
	\texttt{3} & \insertframenumbering[3] & Same as the type \texttt{1} with a progression bar for the current section (in magenta). \\
	\hline
	\end{tabularx}
\end{frame}

\subsection{Continuation Text}
\begin{frame}{\subsecname}
	\begin{itemize}
	\item When continuing a frame, Beamer insert the ``continuation text'' after the title.
	\vfill
	\item To insert the continuation text manually, you should use one of:
		\texttt{{\textbackslash}insertcontinuationtext} \\
		\texttt{{\textbackslash}insertcontinuationwith\{integer\}}
	\item The parameter is the value of the continuation counter to display.
	\vfill
	\item \inlineexample{in the following frame, the macro is used in the title \texttt{{\textbackslash}insertcontinuationtext}.}
	\item \inlineexample{in the second following frame, the macro is used in the title \texttt{{\textbackslash}insertcontinuationwith\{34\}}.}
	\end{itemize}
\end{frame}

\begin{frame}{Example of continuation \insertcontinuationtext}
	The continuation text in the title of this frame is given by the macro \texttt{{\textbackslash}insertcontinuationtext}.
\end{frame}

\begin{frame}{Example of continuation\insertcontinuationwith{34}}
	The continuation text in the title of this frame is given by the macro \texttt{{\textbackslash}insertcontinuationwith\{34\}}.
\end{frame}

%%%%%%%%%%%%%%%%%%%%%%%%%%%%%%%%%%%%%%%%%%%%%%%%%
\section{Sectionning}
\tableofcontentslide[sectionstyle={show/shaded},subsectionstyle={show/show/hide},subsubsectionstyle={hide/hide/hide/hide},sections={1-6}]

\subsection{Table of Contents}

\begin{frame}{Table of Contents/Outline}
	\begin{itemize}
	\item The LE2I-UTBM theme provides a convenient macro to insert a table of contents into a slide: \\
		\texttt{{\textbackslash}tableofcontentsslide[toc options][frame options]}
	\vspace{1em}
	\item It is equivalent to: \\
		\texttt{{\textbackslash}begin\{frame\}[t,frame options]} \\
		\texttt{{\textbackslash}frametitle\{{\textbackslash}translate\{Outline\}\}} \\
		\texttt{{\textbackslash}tableofcontents[toc options]} \\
		\texttt{{\textbackslash}end\{frame\}}
	\vspace{1em}
	\item In addition to the standard options for {{\textbackslash}tableofcontents}, the option \texttt{onlyparts} permits to display the list of the parts, only.
	\end{itemize}
\end{frame}

\subsection{Part Sectioning}

\begin{frame}{Part Sectioning}
	\begin{itemize}
	\item The LE2I-UTBM theme provides specific implementations of the \texttt{{\textbackslash}part} macro: \\
		\texttt{{\textbackslash}part[options]\{title\}} \\
		\texttt{{\textbackslash}part*[options]\{title\}}
	\vfill
	\item Options may be: \begin{itemize}
		\item a string value that is the ``short'' title of the part that is appearing in the table of contents.
		\item the pair \texttt{title=text} to define the ``short'' title.
		\item the pair \texttt{author=text} to define the author of the part.
		\item the pair \texttt{label=id} to define the label of the part.
		\end{itemize}
	\vfill
	\item The stared version of \texttt{{\textbackslash}part} does not add the part in the table of contents.
	\end{itemize}
\end{frame}

\begin{frame}{Changing the prefix in the part's slides}
	\begin{itemize}
	\item By default, each part starts with a slide with only the part's title on, without a prefix such as ``Chapter X''.
	\vfill
	\item The LE2I-UTBM theme provides the following macros to change the part's prefix:
		\begin{itemize}
		\item \texttt{{\textbackslash}insertpartprefix} insert the current part prefix.
		\item \texttt{{\textbackslash}partprefix[counter text]\{text\}} changes the prefix to ``text'' followed by ``counter text''.
		\item \texttt{{\textbackslash}resetpartprefix} resets the prefix to the empty text.
		\end{itemize}
	\end{itemize}
	\vfill
	\begin{example}
		\texttt{{\textbackslash}partprefix[{\textbackslash}arabic\{part\}]\{Chapter\}} \\
		produces: ``Chapter 1'', ``Chapter 2'', \dots
	\end{example}
\end{frame}

\subsection{Appendix}

\begin{frame}{\subsecname}
	\begin{itemize}
	\item The LE2I-UTBM theme supports the appendix part.
	\vspace{1em}
	\item To create the appendix, you must:
		\begin{enumerate}
		\item put the macro \texttt{{\textbackslash}appendix} in your {\TeX} file; or
		\item put the macro \texttt{{\textbackslash}bibliography} in your {\TeX} file.
		\end{enumerate}
	\vspace{1em}
	\item All the slides that are put after the creation of the appendix are assumed to be part of the appendix.
	\item The slides in the appendix are not considered in the total number of slides for the core part of the document (see \texttt{{\textbackslash}inserttotalcoreframenumber}).
	\item The slide numbers in the appendix are roman (not arabic), and the page counter is reset at the begining of the appendix.
	\end{itemize}
\end{frame}

\subsection{Bibliography}

\begin{frame}{Bibliography}
	\begin{itemize}
	\item You are able to include a bibliography in your slides with the two standard {\LaTeX} macros: \\
		\texttt{{\textbackslash}bibliographystyle\{style\}} \\
		\texttt{{\textbackslash}bibliography\{filename\}}
	\vspace{1em}
	\item If you do not call \texttt{{\textbackslash}bibliographystyle}, the default style is \texttt{apalike}.
	\vspace{1em}
	\item When the macro \texttt{{\textbackslash}bibliography} is used, the appendix section is started if it was not already done.
	\item The bibliography slides are \alert{always at the end of the document}. Even if you put slides after the {\textbackslash}bibliography macro.
	\item \emph{The first slide of the bibliography is marked with the label ``\texttt{bibliographyslide}'' for \texttt{{\textbackslash}hyperlink}.}
	\end{itemize}
\end{frame}

%%%%%%%%%%%%%%%%%%%%%%%%%%%%%%%%%%%%%%%%%%%%%%%%%
\section{Special Slides}
\tableofcontentslide[sectionstyle={show/shaded},subsectionstyle={show/show/hide},subsubsectionstyle={hide/hide/hide/hide}]

\subsection{Slide with a Single Figure}

\begin{frame}{Slide with a Single Figure}
	\begin{itemize}
	\item The LE2I-UTBM theme provides a macro that permits to display a picture on the entire slide: \\
			\texttt{{\textbackslash}figureslide[options]\{Title of the slide\}\{file\}}
	\item The size of the picture is adjusted to the slide drawing area.
	\item Example: \texttt{{\textbackslash}figureslide\{XyZ\}\{le2ilogo\}} 
	\end{itemize}
	\vspace{1em}
	\begin{center}
		\fakeslide{XyZ}{
			\put(30,5){\includegraphics[width=.55\linewidth]{le2ilogo}}
		}
	\end{center}
\end{frame}

\begin{frame}{Slide with a Single Figure \insertcontinuationtext}
	\begin{itemize}
	\item Options for \texttt{{\textbackslash}figureslide} are:
		\begin{itemize}
		\item \texttt{width=<length>}, specifies the width of the image.
		\vfill
		\item \texttt{height=<length>}, specifies the height of the image.
		\vfill
		\item \texttt{scale=<float>}, specifies the scaling factor of the image.
		\vfill
		\item \texttt{valign=t|c|b}, specifies the vertical alignment of the image (\texttt{t}: top, \texttt{c}: center, \texttt{b}: bottom).
		\vfill
		\item \texttt{halign=l|c|r}, specifies the horizontal alignment of the image (\texttt{l}: left, \texttt{c}: center, \texttt{r}: right).
		\vfill
		\item \texttt{label=<text>}, specifies the label for the frame.
		\vfill
		\item \texttt{subtitle=<text>}, specifies the subtitle for the frame.
		\end{itemize}
	\end{itemize}
\end{frame}

\figureslide{Example of a Single Figure on a Slide}{le2ilogo}

\begin{frame}{Slide with a Single ANIMATED Figure}
	\begin{itemize}
	\item The AutoLaTeX\footnote{\url{http://www.arakhne.org/autolatex}} tool provides a \LaTeX\ package that enables to include images with layers. Each layers may be displayed in a separate frame by Beamer.
	\vfill
	\item If you have included the AutoLaTeX package, the following macro enables you to display an animated figure on the entire space of a slide: \\
		\texttt{{\textbackslash}animatedfigureslide<framespec>[options]\{title\}\{file\}}
	\item This macro is similar to \texttt{{\textbackslash}figureslide}, except that the given picture must be displayable by the macro \texttt{{\textbackslash}includeanimatedfigure}, provided by AutoLaTeX.
	\end{itemize}
\end{frame}

\begin{frame}{How to Create an ANIMATED Figure?}
	\smaller
	\alertbox{You must use the AutoLaTeX\footnote{\url{http://www.arakhne.org/autolatex}} tool.}
	\begin{enumerate}
	\item Open your favorite SVG editor (Inkscape, etc.).
	\item Create the SVG figure with a layer for each frame.
	\item Put the frame specification into the names of the layers. The frame specification gives the frame numbers for which the SVG layer is displayed. \inlineexample{\texttt{<1-3>} indicates the frames 1 to 3. Do not forget to put the lower-than and upper-than symbols.}
	\item Save and run AutoLaTeX. This tool will create a PDF file for each layer, and a \TeX\ file that is controlling the displaying of the figure.
	\item Include the figure with: \texttt{{\textbackslash}includeanimatedfigure<framespec>[options]\{texfilename\}} or with \texttt{{\textbackslash}animatedfigureslide}.
	\end{enumerate}
\end{frame}

\begin{frame}{Example of ANIMATED Figure}
	\begin{itemize}
	\item Three layers are defined with the following names:
		\begin{itemize}
		\item \texttt{Layer1 <1->} --- display the circle in all the frames, starting from the first.
		\item \texttt{Layer2 <2>} --- display the top yellow star in the second frame.
		\item \texttt{Layer3 <3>} --- display the right yellow star in the third frame.
		\end{itemize}
	\end{itemize}
	% The following code is a copy paste of the \includeanimatedfigure given by AutoLaTeX.
	\begin{center}
		\resizebox{.4\linewidth}{!}{
			\resolvepicturename{layerexample}
			\input{\resolvedfilename}
		}
	\end{center}
\end{frame}

\subsection{Final Slide}

\begin{frame}{A Slide at the End}
	\begin{itemize}
	\item The LE2I-UTBM theme automatically adds a slide at the end of the presentation to avoid ``black'' screen.
	\vfill
	\item The default text on this slide is: ``\translate{Thanks}''.
	\item An other text that is available is: ``\translate{Questions}''.
	\item The third option is to repeat the title slide.
	\vfill
	\item The class options \texttt{thanksslide}, \texttt{questionslide} and \texttt{repeattitleslide} permit to select one of these possibilities.
	\vfill
	\item The macro \texttt{{\textbackslash}finalslidetext\{text\}} may be used to set the text by hand.
	\vfill
	\item \emph{This slide is marked with the label \texttt{finalslide} for \texttt{{\textbackslash}hyperlink}.}
	\vfill
	\item You could display this slide at any moment with: \texttt{{\textbackslash}thanksslide}.
	\end{itemize}
\end{frame}

\subsection{Book Description}

\begin{frame}{Book Description}
	\begin{itemize}
	\item You are able to include a description of a book in your presentation with the macro: \\
		\texttt{{\textbackslash}libraryslide[options]\{picture\}} \\
		\texttt{\{title\}\{authors\}\{How published\}\{ISBN\}}
	\vspace{1em}
	\item The macro creates a slide for a book.
	\vspace{1em}
	\item The \texttt{options} may be composed of pairs of name-value:
		\begin{itemize}
		\item \texttt{frametitle=text}: specifies the title of the frame.
		\item \texttt{subtitle=text}: specifies the subtitle of the book.
		\end{itemize}
	\item If a name is not specified in the options, the ``\texttt{subtitle}'' name is assumed.
	\end{itemize}
\end{frame}

%%%%%%%%%%%%%%%%%%%%%%%%%%%%%%%%%%%%%%%%%%%%%%%%%
\section{Slide Content}
\tableofcontentslide[sectionstyle={show/shaded},subsectionstyle={show/show/hide},subsubsectionstyle={hide/hide/hide/hide},sections={3-}]

\subsection{{\textbackslash}includegraphics}
\tableofcontentslide[sectionstyle={show/shaded},subsectionstyle={show/shaded/hide},subsubsectionstyle={hide/hide/hide/hide},sections={3-}]

\begin{frame}{\textbf{{\textbackslash}includegraphics}}
	\alertbox{The macro \textbf{{\textbackslash}includegraphics} is overridden by the theme.}
	\begin{itemize}
	\vspace{2em}
	\item If you don't specify any optional parameter related to the size of the picture in the document, the \textbf{{\textbackslash}includegraphics} macro will use by default: \\
		\textbf{width={\textbackslash}linewidth}
	\end{itemize}
\end{frame}

\subsection{Drawing}
\tableofcontentslide[sectionstyle={show/shaded},subsectionstyle={show/shaded/hide},subsubsectionstyle={hide/hide/hide/hide},sections={3-}]

\begin{frame}{Put at an Absolute Position}
	\begin{itemize}
	\item If you want to put something at an absolute position in your frame, you could use: \\
		\texttt{{\textbackslash}putat<framespec>(x,y)\{something\}}
	\item \alert{Caution: The added elements are put less deeper than the slide text.}
	\item For putting the elements more deeper than the slide text, use: \\
		\texttt{{\textbackslash}putat*<framespec>(x,y)\{something\}}
	\vspace{1em}
	\item Example: \\
		\texttt{{\textbackslash}putat(100,-20)\{{\textbackslash}color\{red\}\{TESTING\}\}}
	\end{itemize}
	\putat(100,-20){\color{red}{TESTING}}
\end{frame}

\subsection{Boxes}
\tableofcontentslide[sectionstyle={show/shaded},subsectionstyle={show/shaded/hide},subsubsectionstyle={hide/hide/hide/hide},sections={3-}]

\begin{frame}{Alert Boxes}
	The LE2I-UTBM theme defines the a box for alerts: \\
	\begin{itemize}
	\item \texttt{{\textbackslash}alertbox\ensuremath{<}frame\_spec\ensuremath{>}\{this is an alert text\}} \\[.5cm]
		\alertbox{this is an alert text}
		\vspace{1cm}
	\item \texttt{{\textbackslash}alertbox*\ensuremath{<}frame\_spec\ensuremath{>}\{this is an alert text\}} \\[.5cm]
		\alertbox*{this is an alert text}
	\item Note that \texttt{\ensuremath{<}frame\_spec\ensuremath{>}} is optional. It permits to specify the Beamer frame in which the box is displayed.
	\end{itemize}
\end{frame}

\begin{frame}{Inline Example Boxes}
	\begin{small}
	The LE2I-UTBM theme defines the the following macros to put examples in the text (not in a block, as predefined in Beamer):
	\begin{itemize}
	\item \texttt{{\textbackslash}insertexamplelabel} insert the text ``example''.
	\item \texttt{{\textbackslash}insertexampleslabel} insert the text ``examples''.
	\vspace{1em}
	\item \texttt{{\textbackslash}inlineexample\{text\}} insert a example in the text. \\
		Example: \texttt{This is a text followed by an inline example. {\textbackslash}inlineexample\{some text.\}} \\
		This is a text followed by an inline example. \inlineexample{some text.}
	\item \texttt{{\textbackslash}inlineexamples\{text\}} insert examples in the text. \\
		Example: \texttt{This is a text followed by inline examples. {\textbackslash}inlineexamples\{some text.\}} \\
		This is a text followed by inline examples. \inlineexamples{some text.}
	\end{itemize}
	\end{small}
\end{frame}

\begin{frame}[t]{Zooming on a \TeX\ Box}
	\begin{itemize}
	\item Beamer provides the macro \texttt{{\textbackslash}framezoom} to zoom on a part of a frame. \emph{But, it create a new slide and it is difficult to return to the original slide with a single click}.	
	\item The LE2I-UTBM theme provides a new macro for zooming on click.
	\item \texttt{{\textbackslash}zoombox[options]\{box content\}}
		\begin{itemize}
		\item Display the content of the box. When clicked, display the box content after fitting it to the entire screen. When clicked again, the original slide is restore.
		\item \texttt{options} are:
			\begin{enumerate}
			\item \texttt{border=XXpt}: specify the size of the border lines around the box.
			\item \texttt{left=XXpt}: specify the size of the left margin.
			\item \texttt{right=XXpt}: specify the size of the right margin.
			\item \texttt{top=XXpt}: specify the size of the top margin.
			\item \texttt{bottom=XXpt}: specify the size of the bottom margin.
			\item \texttt{margin=XXpt}: specify the size of all of the margins.
			\end{enumerate}
		\end{itemize}
	\end{itemize}
\end{frame}

\begin{frame}[t]{Zooming on a \TeX\ Box \insertcontinuationtext}
	\begin{alertblock}{Caution}
	The macro \texttt{{\textbackslash}zoombox} is working in viewers that are supporting JavaScript (Acrobat Reader...)
	\end{alertblock}
	\vspace{1em}
	\begin{center}
		\zoombox{ZOOMING EXAMPLE}
	\end{center}
\end{frame}

\begin{frame}{Adjusting a Box}
	\begin{itemize}
	\item The LE2I-UTBM style provides the macro \texttt{{\textbackslash}adjustbox} to add margins to a box.
	\vspace{1em}
	\item \texttt{{\textbackslash}adjustbox[options]\{box content\}}
	\vspace{1em}
	\item The options are:
		\begin{itemize}
		\item \texttt{left=XXpt} is the size of the left margin.
		\item \texttt{right=XXpt} is the size of the right margin.
		\item \texttt{top=XXpt} is the size of the top margin.
		\item \texttt{bottom=XXpt} is the size of the bottom margin.
		\item \texttt{size=XXpt} is the size of all of the margins.
		\end{itemize}
	\end{itemize}
\end{frame}

\subsection{Tables, Descriptions and Lists}
\tableofcontentslide[sectionstyle={show/shaded},subsectionstyle={show/shaded/hide},subsubsectionstyle={hide/hide/hide/hide},sections={3-}]

\begin{frame}{Tables with Colors}
	\begin{itemize}
	\item The LE2I-UTBM theme puts colorized borders around tables.
	\item In addition, you could create a table heading with specific colors:
		\begin{itemize}
		\item \texttt{{\textbackslash}tabularheading} to use the heading background.
		\item \texttt{{\textbackslash}chead\{text\}} to define the text of a column heading.
		\end{itemize}
	\end{itemize}
	\begin{example}
		\begin{columns}
			\begin{column}{.6\linewidth}
				\footnotesize
				\texttt{{\textbackslash}begin\{tabular\}\{{\textbar}l{\textbar}l{\textbar}\}{\textbackslash}{\textbackslash}} \\
				\texttt{{\textbackslash}hline} \\
				\texttt{{\textbackslash}tabularheading{\textbackslash}chead\{A\}\&{\textbackslash}chead\{B\}{\textbackslash}{\textbackslash}} \\
				\texttt{{\textbackslash}hline} \\
				\texttt{{\textbackslash}C\&D{\textbackslash}{\textbackslash}} \\
				\texttt{{\textbackslash}end\{tabular\}}
			\end{column}
			\begin{column}{.3\linewidth}
				\begin{tabular}{|l|l|}
					\hline
					\tabularheading\chead{A}&\chead{B} \\
					\hline
					C & D \\
					\hline
				\end{tabular}	
			\end{column}
		\end{columns}
	\end{example}
\end{frame}

\begin{frame}{Enhanced Description}
	The LE2I-UTBM theme provides an enhanced definition of the \texttt{description} environment. \\[.5cm]
	{\smaller
	\texttt{{\textbackslash}begin\{description\}} \\
	\texttt{{\textbackslash}item text1} \\
	\texttt{{\textbackslash}item[Item Name] text2} \\
	\texttt{{\textbackslash}item$<$frame\_spec$>$ text1} \\
	\texttt{{\textbackslash}item$<$frame\_spec$>$[Item Name] text2} \\
	\texttt{{\textbackslash}end\{description\}}} \\[.5cm]
	\begin{description}
	\item text1
	\item[Item Name] text2
	\item<2> text1
	\item<2>[Item Name] text2
	\end{description}
\end{frame}

\begin{frame}{Enhanced Enumeration}
	The LE2I-UTBM theme provides an enhanced definition of the \texttt{enumerate} environment. \\[.5cm]
	{\smaller
	\texttt{{\textbackslash}begin\{enumerate\}[counter format]} \\
	\texttt{{\textbackslash}item text1} \\
	\texttt{{\textbackslash}item[Item Name] text2} \\
	\texttt{{\textbackslash}item$<$frame\_spec$>$ text1} \\
	\texttt{{\textbackslash}item$<$frame\_spec$>$[Item Name] text2} \\
	\texttt{{\textbackslash}end\{enumerate\}}} \\[.5cm]
	\begin{enumerate}
	\item text1
	\item[Item Name] text2
	\item<2> text1
	\item<2>[Item Name] text2
	\end{enumerate}
\end{frame}

\begin{frame}{Enhanced Enumeration \insertcontinuationtext}
	Below, the optional parameter \texttt{counter format} is set to \texttt{"a)"}: \\[.5cm]
	\begin{enumerate}[a)]
	\item text1
	\item[Item Name] text2
	\item<2> text1
	\item<2>[Item Name] text2
	\end{enumerate}
\end{frame}

\begin{frame}{Wrapped Figure on Side of Itemizes}
	\begin{itemize}
	\item For putting a figure on the right side of an itemize environment, the following macros are provided for helping you.
	\item For putting the figure: \\
			\texttt{{\textbackslash}wrapfigure[options]\{figure filename\}}
	\item For putting items on the side of the figure: \\
			\texttt{{\textbackslash}wrapitem[width]\{item text\}}
	\end{itemize}
	\begin{example}\tiny
		\wrapfigure[width=.1\linewidth]{le2ilogo}
		\begin{itemize}
		\wrapitem{item 1 with \texttt{{\textbackslash}wrapitem}.}
		\wrapitem[.85\linewidth]{item 2 with \texttt{{\textbackslash}wrapitem}, item 2 with \texttt{{\textbackslash}wrapitem}, item 2 with \texttt{{\textbackslash}wrapitem}, item 2 with \texttt{{\textbackslash}wrapitem}.}
		\wrapitem{item 3 with \texttt{{\textbackslash}wrapitem}.}
		\wrapitem{item 4 with \texttt{{\textbackslash}wrapitem}.}
		\item{item 5 with \texttt{{\textbackslash}item}, item 5 with \texttt{{\textbackslash}item}, item 5 with \texttt{{\textbackslash}item}, item 5 with \texttt{{\textbackslash}item}, item 5 with \texttt{{\textbackslash}item}, item 5 with \texttt{{\textbackslash}item}.}
		\item item 6 with \texttt{{\textbackslash}item}. 
		\end{itemize}
	\end{example}
\end{frame}

\subsection{Text}
\tableofcontentslide[sectionstyle={show/shaded},subsectionstyle={show/shaded/hide},subsubsectionstyle={hide/hide/hide/hide},sections={3-}]

\begin{frame}{Additional Symbols}
	\begin{itemize}
	\item The LE2I-UTBM theme (re)defines the macros for several symbols:
	\end{itemize}
	\begin{tabularx}{\linewidth}{|l|X|l|X|}
	\hline
	\tabularheading\multicolumn{2}{|c|}{\chead{From LE2I-UTBM theme}} & \multicolumn{2}{c|}{\chead{From \TeX}} \\
	\hline
	\texttt{{\textbackslash}copyright} & \copyright & \texttt{{\textbackslash}copyright} & \textcopyright \\
	\hline
	\texttt{{\textbackslash}trademark} & \trademark & \texttt{{\textbackslash}texttrademark} & \texttrademark \\
	\hline
	\texttt{{\textbackslash}servicemark} & \servicemark & \texttt{{\textbackslash}textservicemark} & \textservicemark \\
	\hline
	\texttt{{\textbackslash}regmark} & \regmark & \texttt{{\textbackslash}textregistered} & \textregistered \\
	\hline
	\texttt{{\textbackslash}checkmark} & \checkmark & \texttt{{\textbackslash}textcheckmark} & \textcheckmark \\
	\hline
	\texttt{{\textbackslash}xmark} & \xmark & & \\
	\hline
	\end{tabularx}
\end{frame}

\begin{frame}{Sizes of the Text}
	\begin{itemize}
	\item The standard Beamer macros for selected the text side are: \\
		{\TINY\texttt{{\textbackslash}TINY}}, {\Tiny\texttt{{\textbackslash}Tiny}}, {\tiny\texttt{{\textbackslash}tiny}}, {\scriptsize\texttt{{\textbackslash}scriptsize}}, {\footnotesize\texttt{{\textbackslash}footnotesize}}, {\small\texttt{{\textbackslash}small}}, {\normalsize\texttt{{\textbackslash}normalsize}}, {\large\texttt{{\textbackslash}large}}, {\Large\texttt{{\textbackslash}Large}}, {\huge\texttt{{\textbackslash}huge}}, {\Huge\texttt{{\textbackslash}Huge}}
	\vspace{1em}
	\item The LE2I-UTBM theme includes two additional macros:
		\begin{itemize}
		\item \texttt{{\textbackslash}smaller} : to decrease the size of the text, and
		\item \texttt{{\textbackslash}larger} : to increase the size of the text.
		\end{itemize}
	\end{itemize}
\end{frame}

\begin{frame}{Emphazing the Text}
	\begin{itemize}
	\item The LE2I-UTBM theme redefines the macro \texttt{{\textbackslash}emph} to display the emphazed text with a color. \\
		Example: \emph{This is an emphazed text.}
	\vspace{2em}
	\item The LE2I-UTBM theme defines the macro \texttt{{\textbackslash}Emph} to display the ``very emphazed'' text with a color. \\
		Example: \Emph{This is a ``very emphazed'' text.}
	\end{itemize}
\end{frame}

\begin{frame}{Underlining the Text}
	\begin{itemize}
	\item The LE2I-UTBM theme redefines the macro \texttt{{\textbackslash}underline} to move the line closer to the text.
	\vspace{2em}
	\item Before: \fakeoldunderline{Example}
	\item After: \underline{Example}
	\end{itemize}
\end{frame}

\begin{frame}{Exponents and Indices}
	\begin{itemize}
	\item The LE2I-UTBM theme (re)defines the macros to put text in exponent or in indice.
	\item The macros \texttt{{\textbackslash}textup} and \texttt{{\textbackslash}textdown} try to add a space after the text, when it is allowed by the typographic rules (it uses the macro \texttt{{\textbackslash}xspace}).
	\end{itemize}
	\begin{tabularx}{\linewidth}{|l|X|l|X|}
	\hline
	\tabularheading\multicolumn{2}{|c|}{\chead{From LE2I-UTBM theme}} & \multicolumn{2}{c|}{\chead{From \TeX}} \\
	\hline
	\texttt{{\textbackslash}textup} & ABC\textup{ABC}D & \texttt{{\textbackslash}textsuperscript} & ABC\textsuperscript{ABC}D \\
	\hline
	\texttt{{\textbackslash}textsubscript} & ABC\textsubscript{ABC}D & - & - \\
	\hline
	\texttt{{\textbackslash}textdown} & ABC\textdown{ABC}D & - & - \\
	\hline
	\end{tabularx}
\end{frame}

\begin{frame}{Quoting the Text}
	\begin{itemize}
	\item The LE2I-UTBM theme provides macros to output localized quotes:
		\begin{description}
		\item[English] \texttt{{\textbackslash}ukquote\{text\}} \\
			Example: \ukquote{text}
		\item[French] \texttt{{\textbackslash}frquote\{text\}} \\
			Example: \frquote{text}
		\item[Latin] \texttt{{\textbackslash}latquote\{text\}} \\
			Example: \latquote{text}
		\end{description}
	\vspace{1em}
	\item The following macros are used by the quote macros:
		\begin{itemize}
		\item \texttt{{\textbackslash}textgravedbl} : \textgravedbl
		\item \texttt{{\textbackslash}textacutedbl} : \textacutedbl
		\item \texttt{{\textbackslash}guillemotleft} : \guillemotleft
		\item \texttt{{\textbackslash}guillemotright} : \guillemotright
		\end{itemize}
	\end{itemize}
\end{frame}

\subsection{Notes}
\tableofcontentslide[sectionstyle={show/shaded},subsectionstyle={show/shaded/hide},subsubsectionstyle={hide/hide/hide/hide},sections={3-}]

\begin{frame}{Footnotes}
	\begin{itemize}
	\item The LE2I-UTBM theme provides two versions of the footnote macro:
		\begin{itemize}
		\item \texttt{{\textbackslash}footnote\{text1\}} shows a footnote\footnote{text1} with a number, and
		\item \texttt{{\textbackslash}footnote*\{text2\}} shows a footnote\footnote*{text2} without a number.
		\end{itemize}
	\vspace{1em}
	\item Additionnally, a footnote with bibliography citation may be added:
		\begin{itemize}
		\item \texttt{{\textbackslash}footcite\{keys\}} shows the given citations in a footnote.
		\end{itemize}
	\end{itemize}
\end{frame}

\begin{frame}{Notes on the Side of the Frame}
	\begin{itemize}
	\item The LE2I-UTBM theme provides a macro to put a text on the side of the frame: \\
		\texttt{{\textbackslash}sidenode\{text\}}
	\vspace{1em}
	\item Example: \texttt{{\textbackslash}sidenote\{text on the side\}}
		\sidenote{text on the side}
	\end{itemize}
\end{frame}

\begin{frame}{Citations on the Side of the Frame}
	\begin{itemize}
	\item The LE2I-UTBM theme provides a macro to put a citation on the side of the frame: \\
		\texttt{{\textbackslash}sidecite\{labels\}}
	\vspace{1em}
	\item It is equivalent to: \\
		\texttt{{\textbackslash}sidenote\{{\textbackslash}cite\{labels\}\}}
	\end{itemize}
\end{frame}

\subsection{Links}
\tableofcontentslide[sectionstyle={show/shaded},subsectionstyle={show/shaded/hide},subsubsectionstyle={hide/hide/hide/hide},sections={3-}]

\begin{frame}[t]{Link to a Video}
	\begin{itemize}
	\item The LE2I-UTBM theme provides a convenient macro to include links to multimedia resources.
	\item \texttt{{\textbackslash}videolink[options]\{resource\_path\}\{img\_path\}}
		\begin{itemize}
		\item Display a picture with a link button. When the user click on the picture, the resource is run (viewed).
		\item \texttt{options} are the options to pass to the \texttt{{\textbackslash}includegraphics} (width...)
		\item \texttt{resource\_path} is the path to the multimedia resource.
		\item \texttt{img\_path} is the path to the picture to display in the slide.
		\end{itemize}
	\end{itemize}
	\begin{center}
		\videolink[width=.3\linewidth]{./documentation.tex}{cnrslogo}
	\end{center}
\end{frame}

\begin{frame}[t]{Link to a Slide with an Picture}
	\begin{itemize}
	\item The LE2I-UTBM theme provides a convenient macro to create links to other slides with a picture in the link.
	\item \texttt{{\textbackslash}picturegoto[options]\{label\}\{img\_path\}}
		\begin{itemize}
		\item Display a picture with a link button. When the user click on the picture, the slide with the given label is displayed.
		\item \texttt{options} are the options to pass to the \texttt{{\textbackslash}includegraphics} (width...)
		\item \texttt{label} is the label of the target slide.
		\item \texttt{img\_path} is the path to the picture to display in the slide.
		\end{itemize}
	\end{itemize}
	\begin{center}
		\picturegoto[width=.3\linewidth]{progressbartypes}{cnrslogo}
	\end{center}
\end{frame}

\begin{frame}{Link to a Slide with an Text}
	\begin{itemize}
	\item The LE2I-UTBM theme provides a convenient macro to create links to other slides with a text in the link.
	\item \texttt{{\textbackslash}textgoto\{label\}\{text\}}
		\begin{itemize}
		\item Display the given text with a link button. When the user click on the text, the slide with the given label is displayed.
		\item \texttt{label} is the label of the target slide.
		\item \texttt{text} is the text to display in the slide.
		\end{itemize}
	\end{itemize}
	\begin{center}
		\textgoto{progressbartypes}{this is a link to another slide}
	\end{center}
\end{frame}


%%%%%%%%%%%%%%%%%%%%%%%%%%%%%%%%%%%%%%%%%%%%%%%%%
\section{Low-level Macros}
\tableofcontentslide[sectionstyle={show/shaded},subsectionstyle={show/show/hide},subsubsectionstyle={hide/hide/hide/hide},sections={3-}]

\subsection{Handlers on Frames}

\begin{frame}{At Begin/End of Frame}
	\begin{itemize}
	\item If you want to do something at the beginning and ending of each frame, you could use: \\
		\texttt{{\textbackslash}AtBeginFrame\{something\}} \\
		\texttt{{\textbackslash}AtEndFrame\{something\}}
	\end{itemize}
\end{frame}

\subsection{Graphic Axes}

\begin{frame}{Graphic Axes}
	\begin{itemize}
	\item The following macro display the graphic axes that may be used for putting something somewhere on the slide: \\
		\texttt{{\textbackslash}graphicaxes}
	\item The following macro draw the axes at the coordinates $(0,0)$: \\
		\texttt{{\textbackslash}showgraphicaxes}
	\item The size of the axes corresponds to 1 unit.
	\end{itemize}
	\begin{example}
		\texttt{{\textbackslash}graphicaxes}\\
		\graphicaxes
	\end{example}
\end{frame}

\subsection{Picture Filename Resolution}

\begin{frame}{Picture Filename Resolution}
	\begin{itemize}
	\item For searching a picture file into the search paths defined with \texttt{{\textbackslash}graphicspath}, you could use the two following macros.
	\item For searching the file: \\
		\texttt{{\textbackslash}resolvepicturename\{partial filename\}}
	\item The previous macro sets the global macro \texttt{{\textbackslash}resolvedfilename} to the full name of the file if it was found; or to \texttt{{\textbackslash}relax} if it was not found.
	\end{itemize}
	\begin{example}
		\texttt{{\textbackslash}graphicspath\{\{./imgs\}\}} \\
		\texttt{{\textbackslash}resolvepicturename\{myfile\}} \\
		\texttt{{\textbackslash}pgfdeclareimage\{myfileid\}\{{\textbackslash}resolvedfilename\}} \\
		\texttt{{\textbackslash}pgfuseimage\{myfileid\}} \\
	\end{example}
\end{frame}

\begin{frame}{Local Definition of the Picture Search Paths}
	\begin{itemize}
	\item You could locally redefine the picture search path in your presentation.
	\item The environment \texttt{graphicspathcontext} permits to override the value of the picture search path inside its content: \\
			\texttt{{\textbackslash}begin\{graphicspathcontext\}\{paths\}} \\
			\texttt{...} \\
			\texttt{{\textbackslash}end\{graphicspathcontext\}}
	\item The provided path must follow the same syntax as the parameter of the \texttt{{\textbackslash}graphicspath} macro.
	\item You could reuse the paths from the enclosing context by putting \texttt{{\textbackslash}old} in the environment's parameter. \\
		\inlineexample{\texttt{\{./path/to/pictures/\}\{./other/path/\},{\textbackslash}old}}
	\end{itemize}
\end{frame}

\end{document}
