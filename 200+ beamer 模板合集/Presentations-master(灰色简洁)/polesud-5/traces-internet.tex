\documentclass[16pt]{beamer}
\usepackage[utf8]{inputenc}

\title{Comme des miettes de pain}

\usetheme{default}

\usepackage[utf8]{inputenc}
\usepackage{amsmath}
\usepackage{amsfonts}
\usepackage{amssymb}
\usepackage{pgf}
\usepackage{color}
\usepackage[frenchb]{babel}
\usepackage{amssymb}
\usepackage{hyperref}

\usefonttheme{default}
\usepackage{DejaVuSans}
%\usepackage[sfdefault]{FiraSans} %% option 'sfdefault' activates Fira Sans as the default text font
\usepackage[T1]{fontenc}
\renewcommand*\oldstylenums[1]{{\firaoldstyle #1}}


\setbeamertemplate{navigation symbols}{} %remove navigation symbols

\author{Cédric Jeanneret (aka \href{https://www.twitter.com/SwissTengu}{@SwissTengu})}
\institute{\href{https://www.ethack.org/}{EthACK.org}}
\date{\today}

\definecolor{linecolor}{HTML}{4d4c4c}

\setbeamercolor{linecolor}{fg=white,bg=linecolor}

\setbeamertemplate{headline} {
	\begin{beamercolorbox}[wd=\paperwidth,dp=8pt,ht=12pt,leftskip=.29cm,rightskip=.29cm]{linecolor}
	\hfill
	\hypersetup{
		colorlinks=true,
		linkcolor=white,
		urlcolor=white,
	}
	\insertinstitute
	\end{beamercolorbox}%
}

\setbeamertemplate{footline}{%
	\begin{beamercolorbox}[wd=\paperwidth,dp=9pt,ht=0.4cm,leftskip=.29cm,rightskip=.3cm]{linecolor}
	\pgfputat{\pgfxy(0.455,-0.315)}{\pgfbox[center,base]{\includegraphics[width=1.5cm]{../common/logo_537.png}}}
	\hfill
	\inserttitle
	\end{beamercolorbox}%
}


\hypersetup{
	colorlinks=true,
	linkcolor=blue,
	urlcolor=blue,
	pdfborderstyle={/S/U/W 1},
	pdfborder=0 0 1,
	linkbordercolor={0 0 0},
	urlbordercolor={0 0 0},
}


\begin{document}

{
\setbeamertemplate{footline}{%
	\begin{beamercolorbox}[wd=\paperwidth,dp=8pt,ht=12pt,leftskip=.29cm,rightskip=.3cm]{linecolor}
	\hfill
	\inserttitle
	\end{beamercolorbox}%
}

% center first slide — not a title, but almost
{
\centering
\begin{frame}

EthACK
\vspace{0.5cm}

The Swiss Privacy Basecamp 
\vspace{0.5cm}

\includegraphics[width=4cm]{../common/logo_537.png}

\end{frame}
}
}

\begin{frame}{EthACK ?}
\begin{itemize}
	\item Éthique
	\item État
	\item ACKnowledgement (reconnaissance)
	\item Hacking (éthique, évidemment)
	\item …
\end{itemize}
\end{frame}

\begin{frame}{Pourquoi ?}
\begin{itemize}
	\item Notre gouvernement ne s'intéresse pas (ou peu) au sujet
	\item Les sociétés privées nous fichent à notre insu
	\item Personne ne sait où sont leurs données, qui les traitent, à quoi elles servent
\end{itemize}
\end{frame}



\begin{frame}
  \titlepage
\end{frame}

\begin{frame}{Hansel et Grethel}
\begin{itemize}
\item App mobile
\item Navigateur web
\end{itemize}
\end{frame}

\begin{frame}
{Cookie : de quoi parle-t-on ?}
Ce sont des fichiers.
\begin{itemize}
\item Déposés sur le client
\item Protection médiocre
\item Durée de vie déterminée par le serveur
\end{itemize}
\end{frame}

\begin{frame}
{Cookie : c'est dépassé}
\begin{itemize}
\item Nouvelles technologies
\item Browser plus faciles à configurer
\item Les cookies sont "connus"
\end{itemize}
\end{frame}

\begin{frame}
{Flash : Objet local partagé}
\begin{itemize}
\item Gestion séparée du navigateur
\item Sécurité bidon
\item Compliqué à gérer
\item Persistant
\end{itemize}
\end{frame}

\begin{frame}
{Flash : c'est dépassé}
\begin{itemize}
\item Abandon (lent)
\item Désactivation possible et simplifiée
\item Mauvaise réputation de Flash (sécurité)
\end{itemize}
\end{frame}

\begin{frame}
{Alors on est sauvé ?}
\centering
Si les cookies et LSO sont dépassés… Non ?
\end{frame}

\begin{frame}
{Le navigateur, une poche trouée}
\begin{itemize}
\item Résolution de l'écran
\item Plugins installés
\item Polices de caractère
\item Localisation
\item Préférences linguistiques
\item … et tant d'autres choses
\end{itemize}
\end{frame}

\begin{frame}
{Javascript, le pickpocket}
\begin{itemize}
\item Exécution côté client
\item Peut renvoyer des infos au serveur
\item Indispensable sur la majorité des sites web
\end{itemize}
\end{frame}

\begin{frame}
{Javascript : presque dépassé}
\begin{itemize}
\item Invasif
\item Simple à désactiver (mais pas pratique)
\item Il y a mieux…
\end{itemize}
\end{frame}

\begin{frame}
{Notion d'empreinte}
\begin{itemize}
\item Récolte des informations système
\item Application d'un algorithme
\item Sortie de l'empreinte
\end{itemize}
\vspace{1cm}
\centering
\textbf{\href{https://panopticlick.eff.org/}{panopticlick.eff.org}}
\end{frame}

\begin{frame}
{Canvas : votre empreinte à votre insu}
\begin{itemize}
\item HTML5
\item Ne peux pas être bloqué
\item Possible manque d'entropie
\item Stocké par les serveurs distants
\end{itemize}
\vspace{1cm}

\href{https://en.wikipedia.org/wiki/Canvas_fingerprinting}{Canvas fingerprinting (Wikipedia)} \newline
\href{https://en.wikipedia.org/wiki/Evercookie}{Evercookie (Wikipedia)}
\end{frame}

\begin{frame}
\centering
\includegraphics[height=\textheight,keepaspectratio]{./screwed.jpg} 
\end{frame}

\begin{frame}
\centering
\includegraphics[height=\textheight,keepaspectratio]{./kidding.jpg} 
\end{frame}

\begin{frame}
{Que faire ?}
\begin{itemize}
\item Hygiène
\item Rélexes
\item Addons
\end{itemize}
\vspace{1cm}
\centering
Basé sur \href{https://www.mozilla.org/fr/firefox/new/}{Firefox}
\end{frame}

\begin{frame}
{Hygiène}
\begin{itemize}
\item Régler son browser
\item Trier les addons et plugins
\item Séparer correctement les profiles
\end{itemize}
\end{frame}

\begin{frame}
{Réglages (condensé rapide)}
\begin{itemize}
\item Page d'accueil
\item Moteur de recherche par défaut
\item Ne pas accepter les cookies tiers
\item Conserver les cookies jusqu'à la fermeture du navigateur
\end{itemize}
\end{frame}

\begin{frame}
{Trier les addons/plugins}
\begin{itemize}
\item Installer uniquement le nécessaire
\item Supprimer quand on n'utilise plus
\end{itemize}
Avantages :
\begin{itemize}
\item Plus rapide
\item Plus léger
\end{itemize}
\end{frame}

\begin{frame}
{Profiles (avancés)}
\begin{itemize}
\item Séparation physique
\item Jetable
\item Addons propre à chaque profile
\end{itemize}
\end{frame}

\begin{frame}
{Réflexes}
\begin{itemize}
\item Employer les signets
\item Ne pas chercher un site que l'on connaît
\item Éviter de cliquer sur tout et n'importe quoi
\item Nettoyer régulièrement les cookies et contenus offline
\end{itemize}
\end{frame}

\begin{frame}
{Addons utiles}
\begin{itemize}
\item \href{https://addons.mozilla.org/en-US/firefox/addon/ublock/}{uBlock}
\item \href{https://addons.mozilla.org/en-us/firefox/addon/ghostery/}{Ghostery} (Attention à Ghostrank)
\item \href{https://addons.mozilla.org/en-us/firefox/addon/lightbeam/}{Lightbeam}
\item \href{https://www.eff.org/https-everywhere}{HTTPS Everywhere} (EFF)
\end{itemize}
\end{frame}

\begin{frame}
{Conclusion}
On laisse plein de traces, dans la vie comme sur le Net. \newline

\vspace{1cm}

Comme on se lave les mains dans la vraie vie,
on nettoie son navigateur après utilisation !
\end{frame}

{
\setbeamertemplate{footline}{%
	\begin{beamercolorbox}[wd=\paperwidth,dp=8pt,ht=12pt,leftskip=.29cm,rightskip=.3cm]{linecolor}
	\hfill
	\inserttitle
	\end{beamercolorbox}%
}
{
\centering
\begin{frame}
{Questions ?}

\href{https://ethack.org/}{https://ethack.org/} \\
\vspace{0.3cm}
\href{https://www.twitter.com/EthACK_org}{@EthACK\_org} on Twitter \\
\vspace{0.3cm}
\href{https://www.facebook.com/ethack.org}{ethack.org} on Facebook

\vspace{0.5cm}

\includegraphics[width=4cm]{../common/logo_537.png}
\end{frame}
}
}

\end{document}
