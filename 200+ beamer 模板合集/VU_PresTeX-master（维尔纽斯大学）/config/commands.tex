\usepackage{multicol}
\usepackage{etex}
\usepackage{stmaryrd}
\usetheme{VU}
%\usefonttheme{
%	default | professionalfonts | serif |
%	structurebold | structureitalicserif |
%	structuresmallcapsserif
%}
\usefonttheme{professionalfonts}
%\useinnertheme{
%	circles | default | inmargin |
%	rectangles | rounded
%}
\useinnertheme{rectangles}
%\useoutertheme{
%	default | infolines | miniframes |
%	shadow | sidebar | smoothbars |
%	smoothtree | split | tree
%}
%\useoutertheme{split}
\setbeamercovered{transparent}

% Without navigation symbols
\beamertemplatenavigationsymbolsempty

%% Formatting
\usepackage{url}
\usepackage{latexsym}			% fancy symbols
\usepackage{color}
%\usepackage{float}

%% Encoding, Symbols
\usepackage[utf8x]{inputenc}
\usepackage{lmodern}
\usepackage{float}
\usepackage{wasysym}
\usepackage{ucs}

%Meta info
%Necessary Information
\author[shortauthor]{Author}
\title{Title}
\subtitle{Subtitle}
%The day of the presentation
\date{\today}

%Optional Information
\subject{subject}
\keywords{keywords}

%Already set
\ifthenelse{\equal{\lang}{lithuanian}}{%
\institute[MIF]{Matematikos ir informatikos fakultetas}}{%
\institute[MIF]{Faculty of Mathematics and Informatics}}
\titlegraphic{\includegraphics[width=13mm,height=13mm]{image/logo}}


%% Hyperref
\usepackage{hyperref}

\makeatletter
\hypersetup{pdftitle={\@title}, pdfauthor={\@author}, linktoc=page, pdfborder={0 0 0 [3 3]}, breaklinks=true, linkbordercolor=unibablueI, menubordercolor=unibablueI, urlbordercolor=unibablueI, citebordercolor=unibablueI, filebordercolor=unibablueI}
\makeatother
%% Define a new 'leo' style for the package that will use a smaller font.
\makeatletter
\def\url@leostyle{%
  \@ifundefined{selectfont}{\def\UrlFont{\sf}}{\def\UrlFont{\small\ttfamily}}}
\makeatother
%% Now actually use the newly defined style.
\urlstyle{leo}

%% Internationalisation
\ifthenelse{\equal{\lang}{lithuanian}}{\usepackage[lithuanian]{babel}%
\usepackage[L7x]{fontenc}}{\usepackage[\lang]{babel}}
%\mode<presentation>{
%% XXX without this the number does not appear
%\AtBeginDocument{\def\figurename{{\scshape Fig.~\thefigure}}}
%}
%%\usepackage{abstract}

%% Mathe und Formeln
\usepackage{calc}
\usepackage{amsmath}
\usepackage{amssymb,amsthm,amsfonts}
\usepackage{dsfont}
\usepackage[nice]{nicefrac}
\usepackage{cancel}  



\usepackage{dirtree}   % tree structures

%%%   For advanced tables   %%
\usepackage{longtable, colortbl}
\usepackage{multicol, multirow}
%
%%%  For graphics %%
\usepackage{graphicx}
\usepackage{pgf}
\usepackage{tikz}
%\usepackage{pgfplots}
\usetikzlibrary{calc,arrows,automata,fit,positioning,trees,backgrounds,shadows,decorations,decorations.markings,decorations.shapes,shapes,patterns,fadings}
\usepackage[font=footnotesize]{subfig}
%\usepackage{fp}
%

%
%%%   For Bibtex with APA Style (American Psychology Association)   %%
%\usepackage[numbers]{natbib}
\usebibitemtemplate{\insertbiblabel}



%\usepackage[numbered,autolinebreaks,useliterate]{mcode}
%\usepackage{verbatim}            % include verbatim  (\verbatiminput) standardpaket
%\usepackage{moreverb} 
%% PseudoCode
%\usepackage{algorithm}
\usepackage{algpseudocode}
%%\usepackage{algorithmicx}
%%\floatname{algorithm}{Algorithmus}
%\algrenewcommand{\algorithmiccomment}[1]{\hskip1em\textcolor{gray!60}{$\rhd$ #1}}
%%\renewcommand{\listalgorithmname}{Algorithmen}
%%\def\algorithmautorefname{Algorithmus}
%
%%% Code Highlighting
%\definecolor{mygray}{gray}{.75}
%\usepackage{listings} 
%\lstset{numbers=left, numberstyle=\tiny, numbersep=6pt} 
%\lstset{language=Python}
%\lstset{classoffset=1, morekeywords={mycontext}, keywordstyle=\color{darkgreen}, classoffset=0, keywordstyle=\color{darkblue}}
%\lstset{basicstyle=\small, showstringspaces=false, commentstyle=\color{mygray}, breaklines=true, captionpos=b}
%\renewcommand{\lstlistingname}{Code-Ausschnitt}
%\renewcommand{\lstlistlistingname}{Code-Ausschnitte}
%\def\lstlistingautorefname{Code-Ausschnitt}


%%%%%%%%%%%%%%%%%%%%%%%%%%%%%%%%%%%%%%%%%%%%%%%%%%%%%%%%%%%%%%%%%%%%%%%%%%%%%%%%%%%%%%%%%%%%
%%%                                   COMMAND SETUP                                       %%
%%%%%%%%%%%%%%%%%%%%%%%%%%%%%%%%%%%%%%%%%%%%%%%%%%%%%%%%%%%%%%%%%%%%%%%%%%%%%%%%%%%%%%%%%%%%

%#1 Relative width
%#2 Filename in the ./image directory
%#3 Caption
%#4 Label for referencing
\newcommand{\image}[4]{
\begin{figure}[H]
\centering
\includegraphics[width=#1\textwidth]{image/#2}
\caption{\footnotesize{#3}}
\label{#4}
\end{figure}
}

% WARNING: Experimantal.
% #1 videofile in the ./image directory
% #2 scalefactor
\newcommand{\video}[2]{%
\includemovie[text={\includegraphics[scale=#2]{image/#1.png}}, autoplay, mouse=true, repeat=1]{}{}{image/#1.swf}}
