% !Mode:: "TeX:UTF-8"
%% 使用 XeLaTeX 编译.
\documentclass[notheorems,mathserif,table]{beamer}
\useoutertheme[height=0.1\textwidth,width=0.15\textwidth,hideothersubsections]{sidebar}
\usecolortheme{whale}      % Outer color themes, 其他选择: whale, seahorse, dolphin . 换一个编译看看有什么不同.
\usecolortheme{orchid}     % Inner color themes, 其他选择: lily, orchid
\useinnertheme[shadow]{rounded} % 对 box 的设置: 圆角、有阴影.
\setbeamercolor{sidebar}{bg=blue!50} % sidebar的颜色, 50%的蓝色.
%\setbeamercolor{background canvas}{bg=blue!9} % 背景色, 9%的蓝色. 去掉下一行, 试一试这个.
\setbeamertemplate{background canvas}[vertical shading][bottom=white,top=structure.fg!25] %%背景色, 上25%的蓝, 过渡到下白.
\usefonttheme{serif}  % 字体. 个人偏好有轮廓的字体. 去掉这个设置编译, 就看到不同了.
\setbeamertemplate{navigation symbols}{}   %% 去掉页面下方默认的导航条.
%%------------------------常用宏包---------------------------------------------------------------------
%%注意, beamer 会默认使用下列宏包: amsthm, graphicx, hyperref, color, xcolor, 等等
\usepackage[UTF8]{ctex}
\hypersetup{CJKbookmarks=true}
\usepackage{subfigure} %%图形或表格并排排列
\usepackage{colortbl,dcolumn}     %% 彩色表格
\graphicspath{{figures/}}         %% 图片路径. 本文的图片都放在这个文件夹里了.

\logo{\includegraphics[height=0.09\textwidth]{whulogo.eps}}%% 武大校徽, MetaPost 文件.

%\renewcommand{\raggedright}{\leftskip=0pt \rightskip=0pt plus 0cm}
%\raggedright %% 中文对齐

\def\hilite<#1>{\temporal<#1>{\color{blue!35}}{\color{magenta}}{\color{blue!75}}}
%% 自定义命令, 源自 beamer_guide. item 逐步显示时, 使已经出现的item、正在显示的item、将要出现的item 呈现不同颜色.

\newcolumntype{H}{>{\columncolor{blue!20}}c!{\vrule}}
\newcolumntype{H}{>{\columncolor{blue!20}}c}  %% 表格设置
%==================================参考文献==============================================================
\newcommand{\upcite}[1]{\textsuperscript{\cite{#1}}}  %自定义命令\upcite, 使参考文献引用以上标出现
\bibliographystyle{plain}
%%%%%%%%%%%%%%%%%%%%%%%%%%%%%%%%%%%%%%重定义字体、字号命令 %%%%%%%%%%%%%%%%%%%%%%%%%%%%%%%%%%%%%%%%%%%%%%

%%%%%%%%%%%%%%%%%%%%%%%%%%%%%%%%%%%%%%%%%%%%%%%%%%%%%%%%%%%%%%%%%%%%%%%%%%%%%%%%%%%%%%%%%%%%%%%%%%%%%%%%
\begin{document}

%%----------------------- Theorems ---------------------------------------------------------------------
\newtheorem{theorem}{定理}
\newtheorem{definition}{定义}
\newtheorem{lemma}{引理}
\newtheorem{corollary}{推论}
\newtheorem{proposition}{性质}
\newtheorem{example}{例}
\newtheorem{remark}{注}

\renewcommand\figurename{\rm 图}
\renewcommand\tablename{\bf 表}
%%----------------------------------------------------------------------------------------------------
    \title{\heiti 粗糙集简介}
    \author[\textcolor{white}{\songti 作者~黄正华}]{\songti 作者~~\textcolor{olive}{黄正华}}
    \institute{\zihao{5} \lishu \textcolor{violet}{武汉大学~~数学与统计学院 }}
    \date{\today}
    \frame{ \titlepage }
%%---------------------------------------------------------------------------------------------------
    \section*{目录}
    \frame{\frametitle{目录}\tableofcontents}
%%===================================================================================================
\section{什么是粗糙集}
%%---------------------------------------------------------------------------------------------------
\begin{frame}\frametitle{What is Rough Set?}
\begin{itemize}
 \item 几个符号:
 \begin{center}
  \setlength{\extrarowheight}{1.5mm}
  %\addtolength{\tabcolsep}{1mm}
  \rowcolors[]{1}{blue!20}{blue!10}
   \begin{tabular}{rl}
      $U$         & 有限论域, $U=\big\{x_1,x_2 , \ldots ,x_n\big\}$.                 \\
      $R$         & 等价关系 (满足自反、对称和传递性).                                 \\
      ${[x]_R}$   & 等价类,  $[x]_R = \big\{ y \in U \bigm|\left( {x,y} \right) \in R\big\}$. \\
      $U/R$       & 等价关系~$R$ 划分论域~$U$, 所得等价类的集合.              \\
    \end{tabular}
 \end{center}\pause
  \item {\heiti 问题: }

  \begin{alertblock}{Question}
     给定~$ X \subseteq U$, 如何用等价类
     \[
     \left[ {x_{i_1} } \right]_R,\, \left[ {x_{i_2} } \right]_R,\, \cdots,\, \left[{x_{i_k} }\right]_R
     \]
    描述表达~$X$?
  \end{alertblock}
\end{itemize}
\end{frame}
%%---------------------------------------------------------------------------------------------------
\begin{frame}
\frametitle{What is Rough Set?}
\begin{itemize}
\hilite<1> \item 给定论域~$U$;  %%               注意 \hilite 的效果.
\hilite<2> \item 用一个等价关系将~$U$ 进行划分;
\hilite<3> \item 给定目标集合~$X$;
\hilite<4> \item $X$ 的下近似~$\underline R X  =\left\{ {x \in U\bigm|\left[ x \right]_R  \subseteq X} \right\}$.
\hilite<5> \item $X$ 的边界域.
\end{itemize}

%    \begin{figure}
%       \begin{center}
%       \multiinclude[graphics={width=\textwidth}]{animation}
%       \end{center}
%    \end{figure}
\end{frame}
%%---------------------------------------------------------------------------------------------------
\begin{frame}\frametitle{粗糙集的定义}
给定~$ X \subseteq U$, 要用~$ {U/R}$
中的元素来描述、表达~$X$, 不一定能精确地进行. 但常常可以用关于~$X$
的一对{下近似、上近似}来界定~$X$, 这导致{粗糙集}概念的产生.\pause

\begin{definition}[{\sc Pawlak}(1982)\upcite{pawlak82}]\label{def:paw}
 设~$R$ 是论域~$U$ 上的等价关系, 对集合~$X\subseteq U$,
偶对~$\left( {\underline R X,\overline R X} \right) $ 称为~$X$ 在近似空间~$ \left( {U,R} \right) $
上的一个粗糙近似, 其中
 {\begin{equation}
 \begin{split}
  \underline R X &=\left\{ {x \in U\bigm|\left[ x \right]_R  \subseteq X} \right\}, \\
  \overline R X &=\left\{ {x \in U\bigm|\left[ x \right]_R  \cap X \ne \varnothing }\right\}.\label{paw1}
 \end{split}
\end{equation}}
 $\underline R X $、$\overline R X$ 分别称为~$X$ 的~$R$ 下近似和~$R$ 上近似.
\end{definition}
\end{frame}
%========================================================================================================
\section[应用举例]{粗糙集应用举例}
%%---------------------------------------------------------------------------------------------------
\begin{frame}\frametitle{一个决策表的例子}\small\zihao{6}
\begin{table} %\caption{\heiti 一个决策表的例子}
%\small\xiaowuhao
\setcounter{subtable}{0}
% \label{tab1}
%   \centering
   \subtable[\zihao{6} 医疗信息决策表 ]{
   \rowcolors[]{1}{blue!20}{blue!10}
    \begin{tabular}{ c!{\vrule}c c c!{\vrule}c  }
  %\hline
  {论域}  &\multicolumn{3}{H}{条\ 件\ 属\ 性 }    &{ 决策属性 } \\
   %\cline{2-4}
  {病人}  & {头痛}  & {肌肉痛}& {体温}  & 流感 \\
  %\hline
  $e_1$            &  是  &  是& 正常  & 否         \\
  $e_2$            &  是  &  是& 高    & 是         \\
  $e_3$            &  是  &  是& 很高  & 是        \\
  $e_4$            &  否  &  是& 正常  & 否        \\
  $e_5$            &  否 &  否 & 高    & 否        \\
  $e_6$            &  否 &  是 & 很高  & 是        \\
  $e_7$            &  否 &  否 & 高    & 是        \\
  $e_8$            &  否 &  是 & 很高  & 否        % \\
 % \hline
 \end{tabular}\label{tab1-a}}
 \qquad\pause
 \subtable[\zihao{6} 数字化表达的决策表]{ %\addtolength{\tabcolsep}{1mm}
 \rowcolors[]{1}{blue!20}{blue!10}
\begin{tabular}{ c!{\vrule}c c c!{\vrule}c }
  % \hline
   {$U$}  &\multicolumn{3}{H}{$C$ }    &{$D$ } \\
  %\hline
  { }  & {$a$}  & {$b$}& {$c$}  & $d$ \\
  %\hline
  1          &  1  &  1 & 1  & 0        \\
  2          &  1  &  1 & 2  & 1         \\
  3          &  1  &  1 & 3  & 1        \\
  4          &  0  &  1 & 1  & 0       \\
  5          &  0  &  0 & 2  & 0       \\
  6          &  0  &  1 & 3  & 1        \\
  7          &  0  &  0 & 2  & 1        \\
  8          &  0  &  1 & 3  & 0        \\
   %\hline
 \end{tabular}\label{tab1-b}}
 \end{table}
\end{frame}
%%---------------------------------------------------------------------------------------------------
\begin{frame}\frametitle{决策表条件属性的区分矩阵}
决策表的区分矩阵如下表所示(由于对称性只给出了其下三角部分).
 \begin{table}
  \centering \addtolength{\tabcolsep}{1mm}
 \begin{tabular}{ccccccccc}
   \hline
        & 1 & 2 & 3 & 4 & 5 & 6 & 7 & 8 \\
   \hline
   1 &         &       &          &       &       &       &       &  \\
   2 & $c$     &       &          &       &       &       &       &  \\
   3 & $c$     & $c $  &          &       &       &       &       &  \\
   4 & $a$     & $a,c$ & $a $     &       &       &       &       &  \\
   5 & $a,b,c$ & $a,b$ & $a,b,c$  & $b,c$ &       &       &       &  \\
   6 & $a,c$   & $a,c$ & $a,c$    & $c $  & $b,c$ &       &       &  \\
   7 & $a,b,c$ & $a,b$ & $a,b,c$  & $b,c$ &       & $b,c$ &       &  \\
   8 & $a,c$   & $a,c$ & $a,c$    & $c$   & $b,c$ &       & $b,c$ &  \\
   \hline
 \end{tabular}\label{dismatrix}
 \end{table}

容易得到条件属性约简为~$\{a,\, c\}$.
\end{frame}
%%---------------------------------------------------------------------------------------------------
\begin{frame}\frametitle{条件属性的约简}
通过属性约简, 决策表简化为如下的形式:
\vspace{-1em}
 \begin{table}\small
  \caption{\heiti 约简的决策表}
  \centering \addtolength{\tabcolsep}{2mm}
\begin{tabular}{ c c c c }
  \hline
   {$U$}   &\multicolumn{2}{c }{$C$ }    &{$D$ } \\
  \cline{2-3}
  { }      & {$a$}   & {$c$}  & $d$ \\
  \hline
  1          &  1  & 1  & 0        \\
  2          &  1  & 2  & 1        \\
  3          &  1  & 3  & 1        \\
  4          &  0  & 1  & 0        \\
  5          &  0  & 2  & 0        \\
  6          &  0  & 3  & 1        \\
  7          &  0  & 2  & 1        \\
  8          &  0  & 3  & 0        \\
  \hline
 \end{tabular}\label{tab1-c}
 \end{table}
由表知, $D/\{d\}=\bigl\{ \{1,4,5,8\},\,\{2,3,6,7\}\bigr\}$;\\
$U/\{a,c\}=\bigl\{ \{1\},\,\{2\},\,\{3\},\,\{4\},\,\{5,7\},\,\{6,8\} \bigr\}$.
\end{frame}
%%---------------------------------------------------------------------------------------------------
\begin{frame}\frametitle{决策规则}
记~$D_0=\{1,4,5,8\},\,D_1=\{2,3,6,7\}$, 则~$\underline R {D_0}=\{1,4\}$,
$\underline R {D_1}=\{2,3\}$.
进而得到确定的决策规则:
\begin{align}
  r_1:   (a,\,1)&\wedge(c,\,1)\longmapsto (d,\,0);\\
  r_2:   (a,\,0)&\wedge(c,\,1)\longmapsto (d,\,0);\\
  r_3:   (a,\,1)&\wedge(c,\,3)\longmapsto (d,\,1);\\
  r_4:   (a,\,1)&\wedge(c,\,2)\longmapsto (d,\,1).
\end{align}
\pause
这样就从无序庞杂的信息中得到为人们提供参考的决策规则:
\begin{align}
     \text{(头痛, 是)且(体温, 正常)}& \longmapsto \text{(流感, 否);}\\
     \text{(头痛, 否)且(体温, 正常)}& \longmapsto \text{(流感, 否);}\\
     \text{(头痛, 是)且(体温, 很高)}& \longmapsto \text{(流感, 是);}\\
     \text{(头痛, 是)且(体温, 高)} & \longmapsto \text{(流感, 是).}
\end{align}
\end{frame}


%%===================================================================================================

\section{参考文献}
%%---------------------------------------------------------------------------------------------------
\begin{frame}\frametitle{参考文献}
\footnotesize
\begin{thebibliography}{10}
{\beamertemplatebookbibitems

\bibitem{zhangwenxiu}
张文修, 吴伟志, 梁吉业, 李德玉.
\newblock {\em 粗糙集理论与方法}.
\newblock 科学出版社, 北京, 2001.}

\bibitem{pawlak82}
Z. Pawlak.
\newblock Rough sets.
\newblock {\em International Journal of Computer Information Science},
  5:341--356, 1982.

\bibitem{VariablePrecision}
W.~Ziarko.
\newblock Variable precision rough set model.
\newblock {\em Journal of Computer and System Sciences}, 46:39--59, 1993.

\bibitem{Katzberg1996ExtensionOfVPRS}
J.~D. Katzberg and W.~Ziarko.
\newblock Variable precision extension of rough sets.
\newblock {\em Fundamenta Informaticae}, 27:155--168, 1996.

\end{thebibliography}
\end{frame}

%%%%%%%%%%%%%%%%%%%%%%%%%%%%%%%%%%%%%%%%%%%%%%%%%%%%%%%%%%%%%%%%%%%%%%%%%%%%%%%%%%%%%%%%%%%%%%%
\begin{frame}
 \begin{center}
{\huge \emph{\textcolor{cyan}{Thank  ~you!}}}\\
\vspace{5mm}\large
\begin{tabular}{ll}
{\sc Author}:  & \textsf{\textcolor{magenta}{H}UANG\ Z\textcolor{magenta}{h}eng-\textcolor{magenta}{h}ua}\\
{\sc Address}: & School of Mathematics \& Statistics\\
               & Wuhan University  \\
               & Wuhan, 430072, China\\
  {\sc Email}: & \href{mailto:huangzh@whu.edu.cn}{\color{blue}huangzh@whu.edu.cn}\\
\end{tabular}
\end{center}
\end{frame}

%%%%%%%%%%%%%%%%%%%%%%%%%%%%%%%%%%%%%%%%%%%%%%%%%%%%%%%%%%%%%%%%%%%%%%%%%%%%%%%%%%%%%%%%%%%%%%

\end{document}
